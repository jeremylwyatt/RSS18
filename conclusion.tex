This paper has presented the first generative-evaluative architecture for dexterous grasping in which both the generative and the evaluative model are learned. Using this generative evaluative architecture the success rate for the top ranked grasp rises from 71.59\% (for GM) to 84.2\% (for GEA) on a simulated test set. We also presented a real data set that is statistically significantly harder than a previously published data set for dexterous grasping. Using this data set the top ranked grasp success rate rose from 57.1\% (GM) to 77.8\% (GEA) and this difference is statistically significant at 0.05.

For future work, we hypothesise that gains can be made by scaling the data set. Second, the failure of grasp optimisation using the evaluative model (EM) needs investigation. Third, we may reformulate the GM to learn from negative examples. Fourth, the EM and GM may be improved by training from autonomously generated data, as in reinforcement learning. The GM is essentially a stochastic factored policy and the EM a learned value function for draws from this policy. 