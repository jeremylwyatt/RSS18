Dexterous grasping of a novel object given a single view is a challenging problem. Generative grasping models learned from demonstration have recently shown some promise. However, while able to produce and rank many candidate grasps for a target object, a generative model gives no guarantee as to how good these grasps will be. Predicting success or failure for a candidate grasp is a challenge, particularly for a novel object with minimal surface recovery, of uncertain and non-trivial mass relative to the maximum grip force, and having unknown friction coefficient(s). This paper addresses this case by combining a data-efficient generative model learner with an evaluative deep network learner that predicts the probability of grasp success. To minimise real training data the evaluative network is trained from a pair of sensor and rigid body simulations. We test on a data set of 49 single-views of real objects that must be grasped, and which are chosen to be challenging for generative approaches. This hybrid architecture shows a grasp success rate of 77.6\% relative to a purely generative dexterous grasp method (57.1\%), while needing only ten demonstrated real grasps.