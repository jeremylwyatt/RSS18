%%%%%%%%%%%%%%%%%%%%%%%%%%%%%%%%%%%%%%%%%%%%%%%%%%%%%%%%%%%%%%%%%%%%%%%%%%
%% Trim Size: 9.75in x 6.5in
%% Text Area: 8in (include Runningheads) x 5in
%% ws-ijhr.tex   :  9-5-2005 
%% Tex file to use with ws-ijhr.cls written in Latex2E.
%% The content, structure, format and layout of this style file is the
%% property of World Scientific Publishing Co. Pte. Ltd.
%% Copyright 1995, 2003 by World Scientific Publishing Co.
%% All rights are reserved.
%%%%%%%%%%%%%%%%%%%%%%%%%%%%%%%%%%%%%%%%%%%%%%%%%%%%%%%%%%%%%%%%%%%%%%%%%%%%
%

%%%%%%%%%%%%% FOR TEMPLATE OF TYPING OUT THE BIBLIOGRAPHY TEXT ONLY %%%%%%%%
\newcounter{myctr}
\def\myitem{\refstepcounter{myctr}\bibfont\parindent0pt\hangindent13pt\themyctr.\enskip}
\def\myhead#1{\vskip10pt\noindent{\bibfont #1:}\vskip4pt}
%%%%%%%%%%%%% FOR TEMPLATE OF TYPING OUT THE BIBLIOGRAPHY TEXT ONLY %%%%%%%%

\documentclass{ws-ijhr}

\hyphenation{op-tical net-works semi-conduc-tor}

\usepackage{graphicx}
\usepackage{multicol}
\usepackage{multirow}
\usepackage[bookmarks=true]{hyperref}
\usepackage{amssymb,amsmath}
\usepackage{subfig}
\usepackage{bbding}

\newcommand{\eq}{Eq.~}
\newcommand{\fig}{Fig.~}
\newcommand{\tab}{Tab.~}
\newcommand{\sect}{Sec.~}

\newcommand{\re}{R}
\newcommand{\rf}{R}
\newcommand{\rl}{L}
\newcommand{\om}{O}
\newcommand{\cm}{M}
\newcommand{\fm}{F}
\newcommand{\hc}{C}
\newcommand{\qd}{Q}
\newcommand{\gpm}{T}
\newcommand{\coll}{W}
\newcommand{\pdf}{\mathbf{pdf}}

\newcommand{\argmax}[1]{\underset{#1}{\operatorname{argmax}}\medspace}

\begin{document}

\markboth{Insert Authors' Names here}{Insert Paper's Title here}

%%%%%%%%%%%%%%%%%%%%% Publisher's Area please ignore %%%%%%%%%%%%%%%
%
\catchline{}{}{}{}{}
%
%%%%%%%%%%%%%%%%%%%%%%%%%%%%%%%%%%%%%%%%%%%%%%%%%%%%%%%%%%%%%%%%%%%%

\title{INSTRUCTIONS FOR TYPESETTING MANUSCRIPTS\\
USING COMPUTER SOFTWARE\footnote{For the title, try not to 
use more than 3 lines. Typeset the title in 10~pt 
Times roman, uppercase and boldface.} }

\author{FIRST AUTHOR\footnote{Typeset names in 8~pt roman,
uppercase. Use the footnote to indicate the 
present or permanent address of the author.}}

\address{University Department, University Name, Address\\
City, State ZIP/Zone,
Country\footnote{State completely without abbreviations, the
affiliation and mailing address, including country. Typeset in 8~pt
Times italic.}\\
first\_author@university.edu}

\author{SECOND AUTHOR}

\address{Group, Laboratory, Address\\
City, State ZIP/Zone, Country\\
second\_author@group.com}

\maketitle

\begin{history}
\received{Day Month Year} %
\revised{Day Month Year}  %
\accepted{Day Month Year} %
%\comby{(xxxxxxxxxx)}
\end{history}

\begin{abstract}
Dexterous grasping of a novel object given a single view is an open problem. This paper makes several contributions to the solution of this problem. First, we present a data set of more than two million simulated dexterous grasps for 294 objects drawn from 20 categories. Second, we present a basic architecture for generation and evaluation of dexterous grasps that is trained on a portion of this data set. Third, we train and evaluate eight different architectural variants on this data set. Finally, we present a real robot implementation and evaluate the most promising methods on data set of real object-pose pairs. We show that our best architectural variant achieves a success rate of 87.8\% on novel objects seen from a single view. This improves on the state of the art. 

%Deep neural networks (DNNs) have been used to learn evaluative models (EMs). Given a grasp and an image, an EM indicates the probability of grasp success. Finding a grasp is then an optimisation problem on this evaluation function, searching over the grasp configuration space. This works well for pinch grasps, but it is an open question whether this approach generalises well to dexterous grasps, where the configuration space is of high dimension. An alternative is to learn a generative model (GM), which maps from images to grasps, subsequently refined by search. Factored GMs scale to high-dimensional configuration spaces, allow data-efficient learning, and generate a wide variety of grasp types, fully exploiting the possibilities of dexterous hands. But they give no guarantee as to the probability of grasp success. This paper shows the benefits of a hybrid architecture. It presents and compares multiple versions of three architectures for dexterous grasping: i) pure GM; ii) pure EM and iii) hybrid GM-EM. Extensive empirical studies were executed in both simulation and on a real robot. These show that hybrid GM-EM outperforms pure GM,  which in turn outperforms pure EM. The best performing GM-EM model achieves 87.7\% on a real robot dexterously grasping 49 novel objects in challenging poses.


%A generative-evaluative learning architecture (GEA) is presented. The generative model (GM) is acquired by data efficient learning from demonstration (LfD), and the evaluative model (EM) is trained in simulation, using grasps proposed by the generative model. When a novel object is presented the generative model proposes grasps, which are ranked by the evaluative model. Experiments show that this GEA architecture improves on a pure generative model (GM). On a challenging set of 49 real objects, GEA has a grasp success rate of 77.6\% relative to a pure GM (57.1\%). It is also shown that grasp optimisation using the EM fails to improve on grasps suggested by the GEA, worsening grasp success rate in simulation by 4.8\% against the baseline. These results provide support for generative-evaluative learning  for dexterous grasping.
\end{abstract}

\keywords{Deep learning; generative-evaluative learning; grasping.}

\section{Introduction}

If robots are to be widely deployed in human populated environments then they must deal with unfamiliar situations. An example is the case of grasping and manipulation. Humans grasp and manipulate hundreds of objects each day, many of which are previously unseen. Yet humans are able to dexterously grasp these novel objects with a rich variety of grasps. In addition, we do so from only a single, brief, view of each object. To operate in our world, dexterous robots must replicate this ability.

This is the motivation for the problem tackled in this paper, which is planning of (i) a dexterous grasp, (ii) for a novel object, (iii) given a single view of that object. We define dexterous as meaning that the robot employs a variety of dexterous grasp types across a set of objects. The combination of constraints (i)-(iii) makes grasp planning hard because surface reconstruction will be partial, yet this cannot be compensated for by estimating pose for a known object model. The novelty of the object, together with incomplete surface reconstruction, and uncertainty about object mass and coefficients of friction, renders infeasible the use of grasp planners which employ classical mechanics to predict grasp quality. Instead, we must employ a learning approach.

This in turn raises the question as to how we architect the learner. Grasp planning comprises two problems: generation and evaluation. Candidate grasps must first be generated according to some distribution conditioned on sensed data. Then each candidate grasp must be evaluated, so as to produce a grasp quality measure (e.g maximum resistable wrench), the probability of grasp success, the likely in-hand slip or rotation, etcetera. These measures are then used to rank grasps so as to select one to execute.
\begin{figure}[t]
\begin{center}
  \includegraphics[width=\columnwidth]{images/contribution.pdf}
  \end{center}
  \caption{The basic architecture of a generative-evaluative learner. When shown a novel object the learned generative model (GM) produces many grasps according to its likelihood model. These are then each evaluated by a learned evaluative model (EM), which predicts the probability of grasp success. The grasps are then re-ranked according to the predicted success probability and the top ranked grasp is executed.}
\label{fig:systemArchitecture}
\end{figure}
Either or both a {\em generative} or {\em evaluative} model may be learned. If only a generative model is learned then evaluation must be carried out using mechanically informed reasoning, which, as we noted, cannot easily be applied to the case of novel objects seen from a single view. If only an evaluative model is learned then grasp generation must proceed by search. This is challenging for true dexterous grasping as the hand may have between nine and twenty actuated degrees of freedom. Thus, for dexterous grasping of novel objects from a single view, it becomes appealing to {\em learn} both the generative and the evaluative model. 

The contributions of this paper are as follows. First, we present a data-set of two million dexterous grasps in simulation that may be used to evaluate dexterous grasping algorithms. Second, we release the source code of the dexterous grasp simulator, which can be used to visualise and re-run the dataset, and gather new data.\footnote{The code and simulated grasp dataset are available \href{https://rusen.github.io/DDG/}{here}. The web page explains how to download the dataset, install physics simulator and re-run the grasps in simulation. The simulator acts as a client alongside a simple http web server to gather new grasp data in a distributed setup (coming soon). Furthermore, we provide a tool to save video-data and replay grasps without running them.} Third, we present a generative-evaluative architecture that combines data efficient learning of the generative model with data intensive learning in simulation of an evaluative model. Fourth, we present multiple variations of the evaluative model. Fifth, we present an extensive evaluation of all these models on our simulated data set. Finally, we compare the two most promising variants on a real robot with a data-set of objects in challenging poses.

The model variants are organised in three dimensions. First, we employ two different generative models (GM1 \cite{kopicki2015ijrr} and GM2 \cite{kopicki2019ijrr}), one of which (GM2) is designed specifically for single view grasping. Second, we employ two different back-bones for the evaluative model based on VGG-16 and ResNet-50. Third, we experiment with two optimisation techniques--gradient ascent (GA) and stochastic annealing (SA)--to search for better grasps using the evaluative model as an objective function.

%The first contribution of this paper is the first generative-evaluative architecture where both generative and evaluative models are learned. Second, three novel evaluative networks, based on existing VGG-16 and ResNet-50 architectures, were proposed. Finally, a grasp simulation dataset containing 2M+ grasps on 295 objects, along with the simulator code will be released to to the community. In order to simulate the wide range of physical variations that real world objects have, a data set is acquired by simulation of  grasps proposed by the generative models on novel objects. We vary the physical characteristics of the objects in each simulated scene (mass, friction) which the robot can not observe. This dataset was used to train three evaluative neural networks. The evaluative models were then used to rank grasps produced by the generative models (GM1 and GM2) in real robot experiments. 

%The paper builds upon two existing generative grasp models that are learned from a small number of demonstrated grasps using the data-efficient method for LfD. Unlike other approaches to deep grasping, which are restricted to power-grasps, the methods are able to perform a wide variety of grasps, including pinch, rim, power, and handle grasps, and to use additional fingers to provide bracing.

The paper is structured as follows. First, we discuss related work. Second, the basic generative model is described in detail and the main features of the extended generative model are sketched. Third, we describe the design of the grasp simulation, the generation of the data set. Fourth, we describe the different architectures employed for the evaluative model. Fifth, we describe the training of the evaluative model, the optimisation variants for the evaluative model and the simulated experimental study. Finally, we present the real robot study.


\section{Background and Related Work}

There are four broad approaches to grasp planning for dexterous hands. First, we may employ analytic mechanics to evaluate grasp quality. Second, we may engineer a mapping from sensing to grasp. Third, we may learn this mapping, such as learning a generative model. Fourth, we may learn a mapping from sensing and a grasp to a grasp success prediction. See \cite{bohg2014data} and  \cite{sahbani2012overview} for recent reviews of data driven and analytic methods respectively.

Analytic approaches use mechanical models to predict grasp outcome \cite{bicchi2000a,Liu2000,Pollard2004,miller2004}. This requires models of both object (mass, mass distribution, shape, and surface friction) and manipulator (kinematics, exertable forces and torques). Several grasp quality metrics can be defined using these~\cite{Ferrari1992,Roa2015,Shimoga1996} under a variety of mechanical assumptions. These have been applied to dexterous grasp planning \cite{Boutselis2014,Gori2014,Hang2014,Rosales2012,Saut2012,ciocarlie2009hand}. The main drawback of analytic approaches is that estimation of object properties is hard. Even a small error in estimated shape, friction or mass will render a grasp unstable \cite{zheng2005a}. There is also evidence that grasp quality metrics are not well correlated with actual grasp success \cite{bekiroglu2011b,kim2013a,goins2014a}. It has been shown that a sampling based grasp planner, using an analytic evaluation model, can achieve grasp success of 53\% on real objects with some parameterized shape uncertainty \cite{li2016dexterous}.

An alternative is learning for robot grasping, which has made steady progress. There are probabilistic machine learning techniques employed for surface estimation for grasping \cite{dragiev2011gaussian}; data efficient methods for learning dexterous grasps from demonstration \cite{ben-amor2012a,kopicki2015ijrr,Osa2018}; logistic regression for classifying grasp features from images \cite{saxena2008a}; extracting generalisable parts for grasping \cite{detry2012a} and for autonomous grasp learning \cite{detry2010a}. Deep learning is a recent approach to grasping. Most work is for two finger grippers e.g. \cite{songiros20}. Approaches either learn an evaluation function for an image-grasp pair \cite{levine16,lenz2015deep,gualtieri2016high,mahler2017dex,pinto2016supersizing,johns2016deep}, learn to predict the grasp parameters \cite{redmon2015real,kumra2017iros} or jointly estimate both \cite{morrison18}. The quantity of real training grasps can be reduced by mixing real and simulated data \cite{bousmalis2017using}. 


\begin{table*}[t]
\centering
\small
\resizebox{\columnwidth}{!}{\begin{tabular}{|l|l|l|l|l|l|l|l|}
\hline
References & \multicolumn{3}{|c|}{Grasp type} & Robot & Clutter & Model & Novel  \\ \cline{2-4}
                   & 2-fing. & $>$2-fing. & $>$2-fing. & results & & free & objects\\ 
                   &            & pow.           & dext. &           &  &        &           \\ \hline
\cite{detry2012a,saxena2008a,detry2010a,lenz2015deep,mahler2017dex,johns2016deep,morrison-RSS-18} & \Checkmark & & & \Checkmark &  & \Checkmark & \Checkmark \\ \hline      
\cite{pinto2016supersizing,bousmalis2017using,levine2017,Gualtieri2016,songiros20} & \Checkmark & & & \Checkmark & \Checkmark & \Checkmark & \Checkmark \\ \hline     
\cite{kappler2015leveraging} &  & \Checkmark & &  &  &  & \Checkmark \\ \hline
\cite{zhou20176dof} &  & \Checkmark & &  &  & \Checkmark & \Checkmark \\ \hline     
\cite{lu2017planning,varley2015generating} &  & \Checkmark & & \Checkmark &  & \Checkmark & \Checkmark \\ \hline    
\cite{ben-amor2012a} &  & &  \Checkmark &  \Checkmark &  &  & \Checkmark \\ \hline    
\cite{veres2017modeling,zhou20176dof,kappler2015leveraging,mandikal2021dexterous} &  & &  \Checkmark &   &  & \Checkmark & \Checkmark \\ \hline 
\cite{kopicki2015ijrr,kopicki2019ijrr,Osa2018} &  & &  \Checkmark &  \Checkmark &  & \Checkmark & \Checkmark \\ \hline 
\cite{arruda2016active} &  & &  \Checkmark &  \Checkmark & \Checkmark & \Checkmark & \Checkmark \\ \hline 
This paper &  & &  \Checkmark &  \Checkmark &  & \Checkmark & \Checkmark \\ \hline 
\end{tabular}}
\caption{Qualitative comparison of grasp learning methods.}
\label{tab:comp-related-work}
\end{table*}

A much smaller number of papers have explored deep learning as a method for dexterous grasping. \cite{lu2017planning,varley2015generating,veres2017modeling,zhou20176dof,kappler2015leveraging}. All of these use simulation to generate the training set for learning. Kappler \cite{kappler2015leveraging} showed the ability of a CNN to predict grasp quality for multi-fingered grasps, but uses complete point clouds as object models and only varies the wrist pose for the pre-grasp position, leaving the finger configurations the same. Varley \cite{varley2015generating} and later Zhou \cite{zhou20176dof} went beyond this by varying the hand pre-shape, and predicting from a single image of the scene. Each of these posed search for the grasp as a pure optimisation problem (using simulated annealing or quasi-Newton methods) on the output of the CNN. They, also, take the approach of learning an evaluative model, and generate candidates for evaluation uninfluenced by prior knowledge. Veres \cite{veres2017modeling}, in contrast, learns a deep generative model. Lu \cite{lu2017planning} learns an evaluative model, and then, given an input image, optimises the inputs that describe the wrist pose and hand pre-shape to this model via gradient ascent, but does not learn a generative model. In addition, the grasps start with a heuristic grasp which is varied within a limited envelope. Mandikal uses reinforcement learning in simulation to learn a wide range of dexterous grasp policies that transfer to unseen objects and which do not require a full mesh \cite{mandikal2021dexterous}. The results are only in simulation and the success rate for unseen objects is 53.1\%.\footnote{Personal correspondence with the authors.} Of the papers on dexterous grasp learning with deep networks only two approaches \cite{varley2015generating,lu2017planning} have been tested on real grasps, with eight and five test objects each, producing success rates of 75\% and 84\% respectively. An key restriction of both of these methods is that they only plan the pre-grasp, not the finger-surface contacts, and are thus limited to power-grasps.

Thus, in each case, either an evaluative model is learned but there is no learned prior over the grasp configuration able to be employed as a generative model; or a generative grasp model is learned, but there is no evaluative model learned to select the grasp. Our technical novelty is thus to bring together a data-efficient method of learning a good generative model with an evaluative model. As with others, we learn the evaluative model from simulation, but the generative model is learned from a small number of demonstrated grasps. Table~\ref{tab:comp-related-work} compares the properties of the learning methods reviewed above against this paper. Most works concern pinch grasping. Of the eight papers on learning methods for dexterous grasping, two \cite{varley2015generating,lu2017planning} are limited to power grasps. Of the remaining six, four have no real robot results \cite{veres2017modeling,zhou20176dof,kappler2015leveraging,mandikal2021dexterous}. Of the remaining four, two we directly build on here, the third being a extension of one of those grasp methods with active vision. Finally, our real robot evaluation is extensive in comparison with competitor works on dexterous grasping, comprising 196 real grasps of 40 different objects.

%n learning to grasp divides into two categories, that we label {\em generative} and {\em evaluative} respectively. We quickly summarise the relationship between this paper and these two.
%
%Generative approaches to grasping include the work of detry et al, kopicki et al, these methods learn distributions of grasps from positive examples. In the case of detry, saxena, these are pinch grasps that are associated with features extracted from monocular or stereo images. Both those pieces of work were restricted to pinch grasps. Both Saxena and Detry require significant training samples, but Detry used real grasps, whereas Saxena used simulation to generate training exemplars. Other approaches use LfD rather than autonomous exploration, as this can significantly reduce the training data required. Peters et al introduced a method for generative grasping by grasp warping, demonstrating an ability to handle new objects of a similar global shape to the training objects. Kopicki et al introduced a factored generative model, showing grasp transfer to novel objects from one example of each grasp type. This is the method than we replicate and build on in our work.
%
%Evaluative approaches to grasping include the work of levine, in which a farm of fourteen robot were used over a two month data gathering period. This data is used to train a neural network that predicts the probability of success of a pinch grasp conditional on an image. In that work the only input is an RGB image, but other approaches to deep grasping tenpas use depth images. However, all these approaches are all data intensive. 
%
%The work that is closest to ours, and the only paper of which we are aware on deep learning applied to dexterous grasping, is that of Hermans et al. 

\section{The Simulated Grasp Data Set}
\label{section:simulation}
%Introductory sentences here
In this section, we describe how we generated a realistic simulated data set for dexterous grasping. This captures variations in both observable (e.g. object pose) and unobservable (e.g. surface friction) parameters.

To generate the training set, a simulated depth image of a scene containing a single unfamiliar object is generated. Using either of the generative models GM1 or GM2, grasps are generated and executed in simulation. The success or failure of each simulated grasp is recorded. Producing a good simulation for evaluating grasps is non-trivial. An important problem is that the data set must capture the natural uncertainty in unobservable variables, such as mass and friction. Since many of these parameters are unobservable we are thus creating a data set such that the grasp policy must work across a range of variations. This is thus a form of {\em domain randomisation}. A similar technique has been employed by \cite{mahler2017dex}, but we extend it from a single grasp quality metric to full rigid body simulation.

\subsection{Features and Constraints of the Virtual Environment}
\label{subsection:environment}

The collected 3D model dataset contains 294 objects from 20 classes, namely, bottles, bowls, cans, boxes, cups, mugs, pans, salt and pepper shakers, plates, forks, spoons, spatulas, knives, teapots, teacups, tennis balls, dustpans, scissors, funnels and jugs (Figure \ref{fig:allObjects}). All objects in the dataset can be grasped using the DLR-II hand, although there are limitations on how some object classes can be approached. For example, teapots and jugs are not easy to grasp except by their handles due being larger than the hand's maximum aperture, while small objects such as salt and pepper shakers can be approached in more creative ways. The number of objects in each class varies from 1 (dustpan) to 25 (bottles). Long/thin objects such as kitchen utensils are placed vertically in a short, heavy stand in order to make them graspable without touching the table. This reflects the real-world scenario, as attempting to grasp a spatula lying on a table would be dangerous for the robotic hand. In total, 250 objects from all 20 classes were allocated for training and validation, while the remaining 44 objects from 19 classes belong to the test set.

We employ MuJoCo \cite{MuJoCo} as the rigid-body simulator. Since MuJoCo requires that objects comprise of convex parts, all 294 objects were decomposed into convex parts using V-HACD algorithm \cite{V-HACD}. The number of sub-parts varies from 2 to 120.

During the scene creation, the object is placed on the virtual table at a pseudo-random pose. Most objects are placed in a canonical upright pose, and only randomly rotated around the gravity axis (akin to a turntable). The objects belonging to the mug and cup classes have fully random 3D rotations, as it is possible to grasp them in almost any setting.

\begin{figure}
  \includegraphics[width=\linewidth]{images/allObjects-small.pdf}
  \caption{A sample of the 294 objects from all 20 object classes.
  \label{fig:allObjects}}
\end{figure}

To achieve domain randomisation, prior distributions for mass, size and frictional coefficient were estimated from real-world data. The properties of simulated objects are sampled from these priors. For each object its mean size, mass and friction coefficient are matched to a real counterpart. For each trial, the size is randomly scaled by a factor in the range [0.9,1.1], while remaining within the grasp aperture of the hand. Object mass is uniformly sampled from a category specific range, estimated from real objects (Table~\ref{fig:weights}). The friction coefficient of each object is sampled from a range of $[0.5, 1]$ in MuJoCo default units, intended to simulate surfaces from low-friction (metal) to high-friction (rubber). This variation is critical to ensuring that the evaluative model will predict the robustness of a grasp to unobservable variations.
\begin{table}[]
\centering
\caption{Mass ranges for each object class (grams).}
\label{fig:weights}
\resizebox{\linewidth}{!}{\begin{tabular}{|l|l|l|l|l|l|l|}
\hline
Bottle & Bowl     & Box     & Can     & Cup    & Fork    & Pan     \\ \hline
30-70  & 50-400   & 50-500  & 200-400 & 30-330 & 40-80   & 150-450 \\ \hline
Plate  & Scissors & Shaker  & Spatula & Spoon  & Teacup  & Teapot  \\ \hline
40-80  & 50-150   & 100-160 & 40-80   & 40-80  & 150-250 & 500-800 \\ \hline
Jug    & Knife    & Mug     & Funnel  & Ball   & Dustpan &         \\ \hline
80-200 & 50-150   & 250-350 & 40-80   & 50-70  & 100-150 &         \\ \hline
\end{tabular}}
\end{table}
 
For depth image simulation the Carmine 1.09 depth sensor installed on the robot is simulated with a modified version of the Blensor Kinect sensor simulator \cite{KinectSimulator}. For each object, we vary the camera orientation and distance from the object, as well as object mass, friction, scale, location and orientation. We add a small three-dimensional positional noise to each point in the sensor output to simulate calibration errors.

A 3D mesh-model of the DLR-II hand has been used in the simulator. There are no kinematic constraints on how the hand may grasp an object, other than collisions with the table. To ensure realism, we use impedance control for the hand.
%The reasons for the most critical of these decisions are now given in slightly more detail. First, in order to create a realistic simulation environment, we chose the MuJoCo \cite{MuJoCo} physics simulator over other simulators (OpenSim, BulletPhysics, ODE, NVIDIA PhysX) for two reasons: 
%\begin{itemize}
%\item MuJoCo uses generalized coordinates and optimization-based contact dynamics, resulting in fewer numerical instabilities,
%\item MuJoCo is optimized for the quality of physics as well as its speed, hence improving the quality of the physics simulation.
%\end{itemize}
\begin{figure}
  \includegraphics[width=\linewidth]{images/decomposition.png}
  \caption{Approximate convex decomposition of some objects in our dataset. Best viewed in colour.}
  \label{fig:objectDecomposition}
\end{figure}

Table \ref{fig:graspperf} shows the success rates of the generated grasps in each class, when attempted with the grasps ranked by the Generative Model (GM1). The sampled grasps perform well on a number of classes including Dustpans, Scissors, Spoons, and Mugs. Some objects can only be grasped in certain ways, i.e. not all 10 training grasps are applicable to all objects.

\begin{table}[]
\centering
\caption{The average and \textbf{top} grasp success rates (\%) of GM1 on simulated data.}
\label{fig:graspperf}
\resizebox{\linewidth}{!}{\begin{tabular}{|l|l|l|l|l|l|l|}
\hline
Bottle & Bowl     & Box     & Can     & Cup    & Fork    & Pan     \\ \hline
35 - \textbf{47} & 26 - \textbf{61}   & 16 - \textbf{30}  & 41 - \textbf{92} & 44 - \textbf{59} & 59 - \textbf{68}   & 37 - \textbf{57} \\ \hline
Plate  & Scissors & Shaker  & Spatula & Spoon  & Teacup  & Teapot  \\ \hline
50 - \textbf{95}  & 62 - \textbf{69}   & 47 - \textbf{53} & 57 - \textbf{65}   & 63 - \textbf{82}  & 48 - \textbf{91} & 26 - \textbf{23} \\ \hline
Jug    & Knife    & Mug     & Funnel  & Ball   & Dustpan &         \\ \hline
24 - \textbf{43} & 58 - \textbf{65}   & 40 - \textbf{80} & 52 - \textbf{65}   & 28 - \textbf{82}  & 60 - \textbf{78} & 45 - \textbf{63} \\ \hline
\end{tabular}}
\end{table}

%\begin{table}[]
%\centering
%\caption{Mass ranges for each object class (grams).}
%\label{fig:weights}
%\resizebox{\linewidth}{!}{\begin{tabular}{|l|l|l|l|l|l|l|}
%\hline
%Bottle & Bowl     & Box     & Can     & Cup    & Fork    & Pan     \\ \hline
%35.5 - \textbf{47.7}\% & 26.4 - \textbf{61.2}\%   & 16.5 - \textbf{30.1}\%  & 41.4 - \textbf{92.6}\% & 44.7 - \textbf{59.9}\% & 59.6 - \textbf{68.1}\%   & 37.9 - %\textbf{57.3}\% \\ \hline
%Plate  & Scissors & Shaker  & Spatula & Spoon  & Teacup  & Teapot  \\ \hline
%50.2 - \textbf{95.5}\%  & 62.7 - \textbf{69.9}\%   & 47.3 - \textbf{53.3}\% & 57.4 - \textbf{65.7}\%   & 63.4 - \textbf{82.4}\%  & 48.2 - \textbf{91.2}\% & 26.9 - %\textbf{23.9}\% \\ \hline
%Jug    & Knife    & Mug     & Funnel  & Ball   & Dustpan &         \\ \hline
%24.9 - \textbf{43.9}\% & 58.3 - \textbf{65.0}\%   & 40.7 - \textbf{80.9}\% & 52.3 - \textbf{65.9}\%   & 28.0 - \textbf{82.8}\%  & 60.1 - \textbf{78.8}\% & 45.8 - %\textbf{63.2}\%        \\ \hline
%\end{tabular}}
%\end{table}

\subsection{Data Collection Methodology}
\label{subsection:dataCollection}

The data set is divided into units called \textit{scenes}, where each scene comprises a single object placed on a table. This object has a specific set of physical parameters, as described below. Many views and grasps are attempted per scene. Below, we specify the time flow of data collection:

\begin{enumerate}
\item A novel instance of an object from the dataset is generated and placed on a virtual table. Variations are applied to object pose, scale, mass, and friction coefficients.
\item A simulated camera takes a depth image $I_s$ of the scene, converted to a point cloud $P_s$. The viewpoint ${elevation}_s$ of the view point is from 30-57 degrees. The ${azimuth}_s$ is sampled from $[0, 2\pi]$. 
\item All points in the point cloud $P_s$ are shifted by a three-dimensional vector sampled from a Gaussian distribution with parameters $\mu=0$ and $\sigma = 0.004$ (unit: meter).
\item Given $P_s$, the chosen generative model (GM1 or GM2) proposes the candidate grasps. For GM1 and GM2, we choose up to 10 and 50 top grasps per each one of the 10 training grasps, respectively.
\item The grasps are applied to the object in simulation. Before the execution of each grasp, we run a collision check with the virtual table (without the object). The grasps that fail this test are marked as \textit{collided}.
\item 19 further simulated depth images are taken from other viewpoints around the object, as explained in step 2. Images with fewer than 250 depth points are discarded. We then sample with replacement from the remaining images and associate each sampled image and viewpoint with a grasp created in step 3.
\item The grasp outcome, trajectory and depth image are stored for each trial. The grasp parameters are converted to the camera frame for the associated view.
\end{enumerate}

%Each candidate grasp $h_i = \{w_0, ..., w_{n}\}$ consists of a series of 10 waypoints along : $w_0$, ..., $w_{n}$. A waypoint $w_k$ is a 27-element vector that specifies full configuration of the hand in joint space: 3 dimensions for 3D coordinates and 4 dimensions for the orientation of the wrist, and 20 parameters specifying each finger joint's activation. 
%After a grasp $h_i$ is generated in world coordinates, the waypoints that belong to the grasp are converted to the camera's frame of reference. 
%The goal of our network architecture is to learn which grasps are more likely to succeed given a point cloud, where both input channels are represented in terms of the camera frame of reference. %This point differentiates us from the work of Levine et al. \cite{Levine1}, where camera coordinates are not used. It should be noted that the possible camera locations in our simulated data covers a larger space, with full circular movement $[0, 2\pi]$ on azimuth and $[30-57]$ range in elevation. Our scenes do not have any distinguishing landmarks such as a bin or robot base, which may aid the network in locating the camera in the scene. 

In each scene $S_i$, a number of depth images are taken $\{I_{ik}\}_{k=0}^{20}$, in the manner explained above. The first image $I_{i0}$ is used to generate grasps, as explained in Section \ref{section:generative}. We typically perform 100-500 grasps per scene. Attaching different views to each grasp instead of the seed image $I_{i0}$ ensures there is more variation in terms of viewpoints, resulting in a richer dataset.

Once a grasp is performed in simulation, it is considered a success if an object is lifted one metre above the table, and held there for two seconds. If the object slips from the hand during lifting or holding, the grasp is a failure. 

\begin{figure}[t]
\includegraphics[width=\columnwidth]{images/frictionweight}
%\includegraphics[width=0.24\textwidth]{images/Pan4_2_HFLW}
%\includegraphics[width=0.24\textwidth]{images/Pan4_2_LFLW}
%\includegraphics[width=0.24\textwidth]{images/Pan4_2_HFHW}
%\includegraphics[width=0.24\textwidth]{images/Pan4_2_LFHW}\\
%%\includegraphics[width=0.96\textwidth]{images/key-to-eval-training}\\
%\includegraphics[width=0.24\textwidth]{images/Pan4_HFLW}
%\includegraphics[width=0.24\textwidth]{images/Pan4_LFLW}
%\includegraphics[width=0.24\textwidth]{images/Pan4_HFHW}
%\includegraphics[width=0.24\textwidth]{images/Pan4_LFHW}
\caption{Creating a data set for robust evaluation. (Top row) The same pinch grasp, executed on the same object, with varying friction and mass parameters. (Bottom row) A more robust power grasp, executed on the same object, with the same variation in friction and mass. \label{fig:evaluative-training}}
\end{figure}

Using this method, we generated a data set (DS1) of 1.28 million simulated grasps using GM1 as the generative model and a data set of 1.136 million additional grasps (DS2) using GM2 \footnote{Visit \href{https://rusen.github.io/DDG/}{https://rusen.github.io/DDG} to download the data.}. Each grasp in DS1-test and DS2 can be replayed in MuJoCo and the sets are decomposed for train, validation and test purposes. We give the dataset statistics in Table~\ref{tab:data}. The ratio of successful grasps in the dataset is less than 50\% for GM1, and is more than 50\% for GM2. In order to have a balanced training set, DS1 and DS2 only contain scenes that have at least one successful grasp. During training, the datasets were balanced by under-sampling the failure cases in DS1-Tr and over-sampling the failure cases for DS2-Tr. No balancing was performed for the validation and test sets.
\begin{table*}[t]
\centering
\caption{Statistics of the simulated data sets.}
\label{tab:data}
\begin{tabular}{|l|l|l|l|l|l|l|l|l|l|l|} \hline
Data set & Generative &  Subset & \# Scenes & Top-grasp & Top-grasp & Top grasp & Total & Total  & Total  & Total \\ 
              & Model         &              &                   &  \# succs  & \# fails       & \% succs  & grasps   & \# succs      & \# fails  & \% succs  \\ \hline
 DS1-Tr & GM1 & Train & 17714 & 10100 & 7614 & 57.0\% & 1,058,430 & 479,941 & 578,489 & 45.3\% \\ \hline
 DS1-V  & GM1 & Validate & 2309 & 1290 & 1019 & 55.9\% & 122,944 & 61,256 & 61,688 & 49,8\% \\ \hline
 DS1-Te & GM1& Test & 1539 & 1070 & 469 & 69.5\% & 99,521 & 48,084 & 51,437 & 48.3\% \\ \hline
 DS2-Tr  & GM2 & Train & 5377 & 3771 & 1606 & 70.1\% & 943,481 & 533,282 & 410,199 & 56.5\% \\ \hline
 DS2-V   & GM2 & Validate & 544 & 378 & 166 & 69.4\% & 68,586 & 39,559 & 29,027 & 57.7\% \\ \hline
 DS2-Te  & GM2 & Test & 988 & 781 & 207 & 79.0\% & 124,137 & 73,836 & 50,301 & 59.5\% \\ \hline
\end{tabular}
\end{table*}



\section{The Generative Evaluative Architecture} \label{section:evaluative}
%In this section we describe how the evaluative models are trained and how we perform gradient descent and stochastic simulated annealing to attempt to improve grasps using the evaluative model.
The grasping system proposed, shown in Figure \ref{fig:systemArchitecture}, consists of a learned generative model and an evaluative model. The generative model is a method that generates a number of candidate grasps given a point cloud, as explained in the previous section. An evaluative model is paired with a generative model in order to estimate a probability of success for each candidate grasp. EM1 and EM3 are VGG-16 based visual models, while EM2 is a modified ResNet-50 architecture. All evaluative models process the visual data and hand trajectory parameters in two separate pathways, and combine them to feed into a third processing block to produce the final success probability. In addition, we will present techniques for grasp optimisation using the EM as the objective function, using both Gradient Ascent (GA) and Simulated Annealing (SA). Finally, we may train each model with either the data set of simulated grasps generated by GM1, by GM2, or by both combined. Table \ref{table:GEBreakdown} shows a the full list of 17 variants we test.

\begin{table}[]
\centering
\begin{tabular}{|l|l|l|l|l|l|l|}
\hline
Variant & GM/  & EM & Opt'  & Training Set \\ 
 & Testset & & Meth' & \\ \hline
V1 & GM1    & - & - & 10 grasps  \\ \hline
V2 & GM2    & - & - & 10 grasps  \\ \hline
V3 & GM1/DS1-Te & EM1 & - & DS1-Tr \\ \hline
V4 & GM1/DS1-Te & EM2 & - & DS1-Tr \\ \hline
V5 & GM1/DS1-Te & EM3 & - & DS1-Tr  \\ \hline
V6 & GM1/DS1-Te & EM1 & - & DS1-Tr + DS2-Tr \\ \hline
V7 & GM1/DS1-Te & EM2 & - & DS1-Tr + DS2-Tr \\ \hline
V8 & GM1/DS1-Te & EM3 & - & DS1-Tr + DS2-Tr \\ \hline
V9 & GM2/DS2-Te & EM1 & - & DS1-Tr + DS2-Tr \\ \hline
V10 & GM2/DS2-Te & EM2 & - & DS1-Tr + DS2-Tr \\ \hline
V11 & GM2/DS2-Te & EM3 & - & DS1-Tr + DS2-Tr \\ \hline
V12 & GM1/DS1-Te & EM3 & GA1 & DS1-Tr + DS2-Tr \\ \hline
V13 & GM1/DS1-Te & EM3 & GA2 & DS1-Tr + DS2-Tr \\ \hline
V14 & GM1/DS1-Te & EM3 & GA3 & DS1-Tr + DS2-Tr \\ \hline
V15 & GM1/DS1-Te & EM3 & SA1 & DS1-Tr + DS2-Tr \\ \hline
V16 & GM1/DS1-Te & EM3 & SA2 & DS1-Tr + DS2-Tr \\ \hline
V17 & GM1/DS1-Te & EM3 & SA3 & DS1-Tr + DS2-Tr \\ \hline
\end{tabular}
\caption{The evaluated combinations of architecture, generative model/test set, training set, and optimisation method (Gradient Ascent (GA) or Stochastic Simulated Annealing (SA).}
\label{table:GEBreakdown}
\end{table}

In this section, the three proposed evaluative model (EM) architectures are explained. The grasp generator models, GM1 and GM2, given in the previous section, require very little training data to train, here being trained from 10 example grasps. %GM1 can generate 500 candidate grasps, ranked according to their estimated likelihoods, within 10 seconds on a 2x Intel Xeon E5-2650 v2 Eight Core 2.6GHz. GM2 takes a 50 seconds to create 250 grasps in the same setting. 
These generative models do not, however, estimate a probability of success for the generated grasps. An evaluative model, which is a Deep Neural Network (DNN), is used specifically for this purpose. DNNs have shown good performance in learning to evaluate grasps using two-finger grippers \cite{levine16,lenz2015deep}. They have also been applied to generating pre-grasps, so as to perform power grasps with dexterous hands \cite{varley2015generating,lu2017planning}.

%The generative approach ignores the global information about the object, such as overall shape and the object category. The success of an executed grasp, however, depends on many contextual factors such as full object shape, mass, mass distribution and surface friction. An evaluative network can indirectly learn to predict grasp success from image data. 
%The data provided to the evaluative network is collected from randomly generated scenes, therefore each scene has a different random combination of the parameters. The primary purpose of the network is to learn robust grasps across different conditions, and this is a complex task. The first challenge is that the kinematic model of the hand is unknown to the evaluative network. It only has access to the parameters that \textit{configure} the hand: the wrist and joint positions. Second, the system is weakly supervised with the grasp result (success or failure), and no further labels are provided.

We tested three evaluative models. The first is based on the VGG-16 network \cite{Simonyan14c}, named Evaluative Model 1 (EM1), and shown in Figure \ref{fig:networkArchitecture2} (a). A version based on the ResNet-50 network, termed EM2, is shown in Figure \ref{fig:networkArchitecture2} (b). Finally, EM3 (Figure \ref{fig:networkArchitecture2} (c) is also based on VGG-16. Regardless of the type, a grasp evaluation network has the functional form $f(I_t, h_t)$, where $I_t$ is a colourised depth image of the object, and $h_t$ contains a series of wrist poses and joint configurations for the hand, converted to the camera's frame of reference. The network's output layer calculates a probability of success for the image-grasp pair $I_t$, $h_t$. The model initially processes the grasp parameters and visual information in separate channels, and combines them to feed into a feedforward pipeline that produces the output. 

%\begin{figure}[h]
%  \includegraphics[width=0.9\linewidth]{images/colourDepth.pdf}
%  \caption[Colourised depth images.]{Colourised depth images. From left to right, the objects are: coke bottle, chocolate box, hand cream, and bowl.}
%\label{fig:colorisedDepth}
%\end{figure}

The depth image is colourised before it is passed as input to the evaluative network. The colourisation process converts the single-channel depth data to a 3-channel RGB image. We first crop the middle $460 \times 460$ section of the $640 \times 480$ depth image, and down-sample it to $224 \times 224$. Then, two more channels of the same dimension are added corresponding to the mean and Gaussian curvatures. %Figure~\ref{fig:colorisedDepth} contains four examples of colourised depth images. 
This procedure both provides a meaningful set of depth features to the network, and makes the input compatible with VGG-16 and ResNet, which require images of size $224 \times 224 \times 3$.

The grasp parameter data $h_t$ consists of 10 trajectory waypoints represented by $27 \times 10 = 270$ floating point numbers, and 10 extra numbers reserved for the grasp type. Each of the 10 training grasps is treated as a different class, and $h_t$ uses the 1-of-N encoding system. Based on the grasp type ([1-10]), the corresponding entry is set to 1, while the rest remain 0. The grasp parameters are converted to the coordinate system of the camera which was used to obtain the corresponding depth image. In EM1 and EM2, the transformed parameters are processed using a fully-connected (FC) layer consisting of 1024 nodes, and the output is \textit{element-wise added} to the visual features. EM3 uses a convolutional approach instead. In all networks, the joint visual features and grasp parameter data are joined in higher layers.

All FC layers have RELU activation functions, except for the output layer, which uses 2-way softmax. The output layer has two nodes, corresponding to the success and failure probabilities of the grasp. A cross-entropy loss is used to train the neural network, as given in \eq\ref{equation:crossentropy}.

\begin{equation}
H_{y'}(y) := - \sum_{i} ({y_i' \log(y_i) + (1-y_i') \log (1-y_i)})
\label{equation:crossentropy}
\end{equation}
where $y_i'$ is the class label of the grasp, which is either 1 (success) or 0 (failure), and $y_i = f(I_i, h_i)$ is is the predicted label of the grasp pair ($I_i$, $h_i)$.

The proposed evaluative models EM1-3 share the common features explained above. The individual models are now introduced below. Only their unique properties are highlighted.

\begin{figure*}[t]
\centering
% \begin{center}
\subfloat[Evaluative Model 1]{%
  \includegraphics[width=0.9\textwidth]{images/networkArchitecture.pdf}
}
% \end{center}

% \begin{center}
\subfloat[Evaluative Model 2]{%
    \includegraphics[width=0.8\textwidth]{images/ResNet.pdf}
}
% \end{center}

% \begin{center}
\subfloat[Evaluative Model 3]{%
    \includegraphics[width=0.8\textwidth]{images/Chaonet_newpic.pdf}
}
% \end{center}

\caption{The three evaluative network architectures presented in this paper. Similar to \cite{Levine1}, the two channels of information (visual data and grasp parameters) are processed in parallel and combined to reach the final decision in each model. RELU activations are used throughout the models, except for the final softmax layers. A final softmax layer has grasp success and and failure nodes, and learns to predict the success probability of a grasp. (a) A VGG-16 based model, where the first 13 convolutional layers of VGG-16 are frozen. (b) ResNet50-based \cite{HeZRS15} network. The first four blocks are used for feature extraction, and the rest of the network is used to learn joint features. (c) Second model based on the VGG-16 architecture. In EM3, the information pathways are joined via concatenation, not addition.
\label{fig:networkArchitecture2}}
\end{figure*}

\subsection{Evaluative Model 1 (EM1)}

Figure~\ref{fig:networkArchitecture2} (a) shows the architecture of the first proposed evaluative network. The colourised depth image is processed with the VGG-16 network \cite{Simonyan14c} to obtain the high-level image features. The VGG-16 network is initialised with the weights obtained from ImageNet training. Only the last three convolutional layers are trained, and the first 13 layers remain frozen. This decision was made in order to speed up training and reduce overfitting.

The grasp parameters and image features pass through fully-connected layers with 1024 hidden nodes (FC-1024) layers in order to obtain two feature vectors of length 1024. The features are combined using the element-wise addition operation, and are further processed using 4 FC-1024 layers. Similarly with \cite{Levine1}, the features are combined using addition and not concatenation. This follows the observation that addition yielded a marginally better performance in the experiments. Furthermore, concatenation and addition can be considered as interchangeable operations when combining different information pathways in deep networks \cite{dumoulin2018feature-wise}. The final FC-1024 layers form the associations between the visual features and hand parameters, and contain most of the trainable parameters in the network. 

\subsection{Evaluative Model 2 (EM2)}

% \begin{figure}[!ht]
%   \includegraphics[width=\textwidth]{images/ResNet.pdf}
%   \caption[The ResNet-based evaluative deep neural network architecture.]{The ResNet-based evaluative deep neural network architecture (EM2). Spatial tiling is used to repeat the grasp parameters before they join the image processing pathway. This network requires fewer FC layers due to the earlier marriage of information channels.}
% \label{fig:ResNet}
% \end{figure}

The second evaluative model, termed EM2 (Figure~\ref{fig:networkArchitecture2} (b)), uses the ResNet-50 network in order to obtain the high-level image features. The ResNet-50 architecture is initialised with the weights obtained from ImageNet training. One exception is the last convolution block conv\_5x, which is initialised randomly. In the EM2 architecture, ResNet-50 network is broken down into two parts: the first 4 convolutional blocks are used to extract the visual features. The final convolutional block, which has 9 convolutional layers, combines the image features and grasp parameters. Similarly with EM1, element-wise addition joins the two channels of information. Spatial tiling is used to convert the processed grasp parameters, a vector of size $1024$, to a matrix of size $14 \times 14 \times 1024$. Because the last convolutional block conv\_5x processes combined information, this network is designed with only 2 FC layers with 64 hidden nodes each. The output layer is the same as EM1. 

\subsection{Evaluative Model 3 (EM3)}

This model, similar to EM1 (Figure~\ref{fig:networkArchitecture2} (c)), uses VGG-16 as the base visual model. All 16 layers of VGG-16 are trained. The hand trajectory parameters pass through a feature extraction network before being concatenated with the visual features. Two fully connected layers of higher capacity than EM1-2, contain 4096 hidden nodes each and join the two pathways. A final 2-node FC layer with softmax activation is added to obtain the final decision.

EM3, contrary to EM1 and EM2, uses convolutional layers for feature extraction when processing input grasp trajectories. The trajectory processing pathway is identical to the VGG-16 architecture in that it contains 5 blocks, consisting of a total 13 convolutional layers. The sizes under the blocks are input dimensions. Global Average Pooling (GAP) is performed to obtain 512 features coming from both sides, and they are concatenated to get the combined feature vector of 1024, which further runs through two FC-4096 layers.

All models were trained and tested on simulated data. In addition, EM2 and EM3 were tested on the real robot setup. The next section focuses on the collection of the training data.

\subsection{EM training methodology}
Variants V3-V5 were trained using DS1-Tr. Variants V6-V17 were trained using the combined data set from DS1-Tr and DS2-Tr. The Gradient Descent(GD) optimiser was employed with starting learning rate of 0.01, a dropout rate of 0.3, and early stopping. We halve the learning rate every 5 epochs during training.

\subsection{Grasp optimisation using the EM}

So far we have considered only Generative-Evaluative architectures where the Evaluative Model merely ranks the grasp proposals. As proposed by Lu et al. \cite{lu2017planning} we may also use the EM to improve grasp proposals. This boils down to searching the grasp space driven by the EM as the objective function. This may be by gradient ascent or simulated annealing. The methods V12-17 use V8 as the objective function, hence V8 should be treated as the baseline. We employed both gradient descent based optimisation and simulated annealing.

\subsubsection{Gradient based optimisation}
Lu et al. \cite{lu2017planning} proposed gradient ascent (GA), modifying the grasp parameters input to the EM with respect to the output predicted success probability. They initialised with a heuristically selected pre-grasp. We initialise with the highest ranked grasp according to the EM. We investigated three variants:
\begin{itemize}
\item GA1: Shifts the position of the all waypoints in the grasp trajectory equally. The gradient is the average gradient of the EM output across all 10 waypoints
\item GA2: Tunes the hand configuration by tuning the angle of each finger joint. Every finger joint at each waypoint is treated independently.
\item GA3: Performs GA1 and GA2 simultaneously.
\end{itemize}

\subsubsection{Simulated annealing based optimisation}
Gradient based optimisation is sensitive to the quality of gradient estimates derived from the model. Simulated annealing (SA) based optimisation is more robust to such noise. Therefore, three optimisation routines were implemented using SA:
\begin{itemize}
\item SA1: Shifts the positions of the all waypoints in the grasp trajectory equally. Moves are drawn from a three-dimensional Gaussian with $\mu=0$ and $\sigma=0.001$. 
\item SA2: Scales the angles of the finger joints in the final grasp pose with a single scaling parameter drawn from a Gaussian with $\mu=1$ and $\sigma=0.001$. The initial finger joint angles remain fixed and joint angles of the intermediate waypoints are linearly interpolated. 
\item SA3: Performs SA1 and SA2 simultaneously.
\end{itemize}



\section{Simulation Analysis}
\label{section:simulationAnalysis}
This section presents a simulation analysis of the evaluative network. Recall that the generative-evaluative architecture (GEA) comprises both the generative model (GM) and the evaluative model (EM). We can evaluate aspects of these separately. After training, the EM was used to predict grasp outcomes in the test set. This comprised 1,241 scenes with 76,213 grasps. Of these, 40,243 succeeded and 35,970 failed. Our analysis is given in Figure~\ref{fig:predictions}. The sensitivity is 0.84 and the specificity 0.71. The F1-score is 0.802.
\begin{table}[b]
\centering
\caption{Confusion matrix for prediction on simulated data.}
\label{fig:predictions}
\begin{tabular}{|c|c|c|c|c|c|}
\hline
 & & \multicolumn{4}{c|}{Prediction} \\ \cline{3-6}
      & & \multicolumn{2}{c|}{\#} & \multicolumn{2}{c|}{\%} \\ \cline{3-6}
  &  & Succ         & Fail         & Succ         & Fail         \\ \hline
\multirow{ 2}{*}{Ground Truth} \newline & Succ & 33890      & 6353       & 84\%     & 16\%       \\ \cline{2-6}
 &Fail & 10339      &  25631    & 29\%     &  71\%   \\ \hline
\end{tabular}
\end{table}
% In this context, precision is the fraction of correctly labeled grasps among those predicted to be of a certain class (success or failure). Recall stands for the fraction of relevant grasps that have been identified correctly among all grasps that belong to that class. The results show a high recall rate for successful grasps, and there are relatively more false positives than false negatives. This necessitates pairing our evaluative neural network with a generative model rather than a random grasp generator, which would likely result in very low quality grasps and consequently, more false positives. 

%To test our generative-evaluative learning architecture we compared the grasp it proposes to the grasp proposed by the generative learner alone. Since \citet{kopicki2015ijrr} showed a 77.7\% success rate with the original generative algorithm we generated a new test set that contained both more challenging objects and placed them in challenging poses. The difficulty single-view grasping with a depth camera depends greatly on the pose of the object relative to the camera. The set comprised 40 test objects (Figure~\ref{fig:real-objects}) and another six training objects. The training objects were used by the human to demonstrate ten example grasps (Figure~\ref{fig:generative-training}). The 40 test objects were used to generate 49 object-pose pairs. From the 40 objects, 35 belonged to object classes in the simulation dataset, while the remaining five do not. 

\begin{figure}[t]
  \includegraphics[width=\linewidth]{images/successvsranking.pdf}
  \caption{Ground truth grasp success probability (in simulation) vs. grasp ranking (by GM).}
  \label{fig:successvsranking}
\end{figure}

We compared the EM and GM rankings (Figure~\ref{fig:successvsranking}). The x-axis shows the ranking. The y-axis shows the average actual success rate over all scenes (1,241 test, 7,311 training). When ranked by the EM, the grasp success probability falls nearly monotonically, as is desirable. On the other hand, the likelihood-based ranking of GM results in many good grasps being low-ranked. We also wish to know whether the grasps recommended by the EM and the GM have different grasp success rates. The success rates of the top-ranked grasps are 71.59\% (GM) and  84.2\% (EM).

We also investigated using the EM to improve grasps directly. Lu et al. \cite{lu2017planning} proposed gradient ascent on the input grasp parameters to the EM with respect to the predicted success probability. They initialised with a heuristic grasp. We initialise with the best grasp proposed by the GEA. We ran 20 and 100 epochs of gradient ascent on the EM network inputs. Although predicted success rate rises, the success rate in simulation declines. After 20 epochs it is 1.3\% lower and after 100 epochs it is 4.8\% lower than the actual success rate of the initial grasp. This suggests that optimising dexterous grasps by the EM is non-trivial, perhaps because of the high-dimensionality of the grasp space. We speculate that initialising with a random grasp would be even worse.

%A pure generative model architecture (GM) and the generative-evaluative architecture (GEA) were evaluated using a paired trials methodology. Each was presented with the same object-pose combinations. Each architecture generated a ranked list of grasps, and the highest ranked grasp was executed. The highest-ranked grasp based on the predicted success probability of the network is performed on each scene. A grasp was deemed successful if, when lifted for five seconds, the object then remained stable in the hand for a further five seconds before being automatically released. The success rate for GM was 57.1\% and for GEA it was 77.6\%. The successes and failures for each method were recorded and are summarised in Table~\ref{tab:robot-results}. A two-tailed McNemar test, for the difference between success rates for paired comparison data, was performed and the difference between the two algorithms has a $p$-value of 0.0442, and so is statistically significant. A selection of grasps where the two methods performed differently are shown in Figure~\ref{fig:successfail}.

% OLD TABLE
%\begin{table}
%\begin{center}
%\caption{Results of the real robot paired comparison trial.}
%\begin{tabular}{|c|c|c|c|}  \hline 
%          &                & \multicolumn{2}{ c |}{ GM} \\ \hline
%          &                & \# succs & \# fails  \\  \hline
 %GEA  & \# succs &  23 &  15  \\
 %         & \# fails    &  5   &   6   \\ \hline
%\end{tabular}
%\end{center}
%\label{tab:robot-results}
%\end{table}

%Training parameters for network. Training of example grasps for learning from demonstration. Creation of real test data set. Paired comparisons methodology with vanilla LFD algorithm (pose + object + camera view).
%
%The actual grasping tests have been performed on the real robot. 

\section{Extended Simulation Analysis}
\label{section:extendedSimulationAnalysis}

\noindent
In this section, we perform a deep dive into the results of V11, specifically the evaluative model EM3 and its predictions on the DS2-Test set containing 124,137 grasps. 

\subsection{Success rates of grasp types}
\noindent

\begin{figure}
\centering 
\includegraphics[width=0.6\columnwidth]{images/post-analysis/Grasp_type_vs_success_prob.png}
\caption{Grasp type vs. success probability.}
\label{fig:post2}
\end{figure}

Not all grasps are created equal. On average, some grasp types are more successful than others. We found that grasp types where all five fingers have contact with the object, such as Rim, Rim Side, or Power Cube are generally more successful than the ones where only 2-3 fingers touch the object, e.g. Pinch and Handle variations. Figure~\ref{fig:post2} compares the average grasp success rates in the DS2-Te (GM2 Test) dataset, which illustrates this phenomenon. We believe this is due to the increased tolerance for finger positioning errors when more contacts are involved. This would be consistent with the dynamics of human grasping of objects. 

\subsection{Are EM3 outputs calibrated?}
\noindent

An interesting question is whether the grasp success probabilities EM3 yields are \textit{calibrated}, i.e. whether they correlate with actual grasp success rates. Calibrated classifiers are desirable, because they are useful in making decisions by incorporating the uncertainty of predictions into account. Figure~\ref{fig:calibrate}(All) plots a histogram of successful and unsuccessful grasps, ordered by their predicted grasp success probabilities. As the blue line shows, the actual grasp success rate in each bin increases near-linearly with the predicted probability. For many grasp types, including Pinch Bottom, Power Cube, Rim, and Power Edge, the grasp success probabilities EM3 yields appear to be correlated with the actual success rates of grasps. This suggests that V11 performance in the real robot experiments could be further improved by employing multiple views of the same object until a grasp is found where the network is highly confident that the grasp will succeed. Applying such an algorithm is beyond the scope of this paper, where the restrictions limits us to one view per object to generate grasps.

\begin{figure}
\centering
\includegraphics[width=0.98\columnwidth]{images/post-analysis/V11_pred_success_vs_success.png}
\caption{Predicted success probability vs actual success probability for V11. The red bars indicate failures, and green bars show successes. The grasps are ordered along the horizontal axis by predicted success probabilities by EM3. The blue line shows the (actual) average success rate of grasps in every bin, while the vertical length of the shaded blue region around each point on the line correlates with the standard deviation of the success rates in nearby bins. Note that when there are not enough grasps to estimate rates, as is the case for Handle and Power Tube grasps, the uncertainty of the estimates increase significantly.}
\label{fig:calibrate}
\end{figure}

\subsection{Why is re-ranking with an evaluative model improving the performance?}
\noindent

V11, which combines the generative model GM2 with the evaluative model EM3, yields a substantial improvement over GM2's ranking, improving GM2's simulation top-grasp success rate from 79.05\% to 90.49\%, and real-world performance from 81.6\% to 87.8\%. We hypothesise that two main factors could contribute to this result. First, EM3 could be assigning high predicted success probabilities to grasp types that are generally successful, and low probabilities to those grasp types that fail often. Effectively, this would mean that a ranking based on EM3's predicted probabilities would simply be favouring grasp types that are, on average, more successful than others. Second, within each grasp type, EM3 may be learning which approach trajectories and hand/finger positions are more likely to be successful with respect to the object. The second factor, if true, is more interesting. Such evidence would imply that the network is learning to associate the object's geometry with the approach trajectory and configuration. Below, we look for evidence to support both hypotheses.

\subsubsection{How does an evaluative model rank grasp types?}
\noindent

%\begin{figure}
%\centering
%\includegraphics[width=0.8\columnwidth]{images/post-analysis/Grasp_type_distributions_all.png}
%\caption{Comparison of grasp type distributions per ranking of random, V2 and V11 rankings.}
%\label{fig:post6}
%\end{figure}

TODO: RE-PLOTTING THE FIGURE

Figure~\ref{fig:post6} compares 3 different rankings (random, V2, V11), and plots the distribution of grasp types for each position in all rankings. The random ranking which are based on randomly assigned success probabilities, predictably, uniformly places all grasps at all positions, regardless of their type. V2, which is based on the grasp success likelihoods generated by GM2, favours successful grasp types such as Rim variations over unsuccessful ones such as Handle grasps. V11, based on the evaluative model EM3, acts similarly to V2. Interestingly, V2 and V11 appear to have similar ranking patterns, despite calculating the grasp success probabilities in entirely different ways. 


%\begin{figure}
%\centering
%\includegraphics[width=0.8\columnwidth]{images/post-analysis/[3] V2_vs_V11_ranking_vs_success_fail.png}
%\caption{Comparison of successful (green) and unsuccessful (red) grasps per ranking for V2 and V11.}
%\label{fig:post3}
%\end{figure}

\subsubsection{Does EM3 rank grasps within a grasp type?}
\noindent

The second hypothesis we have is that EM3 is learning to rank grasps within each grasp type, favouring grasp trajectories that are more likely to be successful over others. In order to test this hypothesis, we compared three different strategies above: random, V2, V11. In Figure~\ref{fig:post5}, we compare the the quality of the ranking of grasps within each grasp type. In order to compare rankings, a model's output score is used as the classification score in a binary classification problem (failure/success). The metric used to compare different models is the area under Receiver Operating Characteristics (ROC) curve. This metric is calculated for each scene, and averaged across all scenes in the test set. The left-most plot in Figure~\ref{fig:post5} shows that the overall ranking quality of V11 is better than V2 and random predictions across grasp types. Furthermore, V11 has the best overall within-type ranking for all grasp types except for Power Tube. This observation suggests that V11 not only leans on the grasp type information, but also learns to associate the object's geometry with the grasp trajectories to determine whether a grasp will be successful.

\begin{figure}
\centering
\includegraphics[width=0.999\columnwidth]{images/post-analysis/Ranking_quality_mean_AUC.png}
\caption{Within-type grasp ranking quality comparison of random, V2 and V11 predictions according to the Area-Under-Curve of Receiver Operatig Characteristics (AUC-ROC) metric. Higher score is better. V11 outperforms random and V2 except for Power Tube.}
\label{fig:post5}
\end{figure}

%\begin{figure}
%\centering
%\includegraphics[width=0.8\columnwidth]{images/post-analysis/Average_success_rate_in_scenes_based_on_top_grasp_success.png}
%\caption{Average success rate in scenes vs top grasp success.}
%\label{fig:post7}
%\end{figure}

%\begin{figure}
%\centering
%\includegraphics[width=0.8\columnwidth]{images/post-analysis/V11_success_prediction_for_top-ranked_grasp.png}
%\caption{Histogram of success rate of a top-ranked grasp vs its probability of success as estimated by V11.}
%\label{fig:post8}
%\end{figure}

%\begin{figure}
%\centering
%\includegraphics[width=0.8\columnwidth]{images/post-analysis/number_of_generated_grasps_vs_success_rate.png}
%\caption{Correlation of success rate with number of generated grasps per scene per grasp type.}
%\label{fig:post10}
%\end{figure}


%We compared the EM and GM rankings (Figure~\ref{fig:successvsranking}). The x-axis shows the ranking. The y-axis shows the average actual success rate over all scenes (1,241 test, 7,311 training). When ranked by the EM, the grasp success probability falls nearly monotonically, as is desirable. On the other hand, the likelihood-based ranking of GM results in many good grasps being low-ranked. We also wish to know whether the grasps recommended by the EM and the GM have different grasp success rates. The success rates of the top-ranked grasps are 71.59\% (GM) and  84.2\% (EM).

%A pure generative model architecture (GM) and the generative-evaluative architecture (GEA) were evaluated using a paired trials methodology. Each was presented with the same object-pose combinations. Each architecture generated a ranked list of grasps, and the highest ranked grasp was executed. The highest-ranked grasp based on the predicted success probability of the network is performed on each scene. A grasp was deemed successful if, when lifted for five seconds, the object then remained stable in the hand for a further five seconds before being automatically released. The success rate for GM was 57.1\% and for GEA it was 77.6\%. The successes and failures for each method were recorded and are summarised in Table~\ref{tab:robot-results}. A two-tailed McNemar test, for the difference between success rates for paired comparison data, was performed and the difference between the two algorithms has a $p$-value of 0.0442, and so is statistically significant. A selection of grasps where the two methods performed differently are shown in Figure~\ref{fig:successfail}.

% OLD TABLE
%\begin{table}
%\begin{center}
%\caption{Results of the real robot paired comparison trial.}
%\begin{tabular}{|c|c|c|c|}  \hline 
%          &                & \multicolumn{2}{ c |}{ GM} \\ \hline
%          &                & \# succs & \# fails  \\  \hline
 %GEA  & \# succs &  23 &  15  \\
 %         & \# fails    &  5   &   6   \\ \hline
%\end{tabular}
%\end{center}
%\label{tab:robot-results}
%\end{table}

%Training parameters for network. Training of example grasps for learning from demonstration. Creation of real test data set. Paired comparisons methodology with vanilla LFD algorithm (pose + object + camera view).
%
%The actual grasping tests have been performed on the real robot. 

\section*{Acknowledgements}

The authors gratefully acknowledge funding from the European Commission Funded project, PaCMan FP7-IST-600918.

\appendix

\section{Data Efficient Learning of a Generative Grasp Model from Demonstration}

%Describe, in new words, the method from IJRR.
This section describes the generative model learning. We employ a method \cite{kopicki2015ijrr}, which learns a generative model of a dexterous grasp from a demonstration (LfD). That paper posed it as the problem of learning a factored probabilistic model. The method is split into a model learning phase, a model transfer phase, and the grasp generation phase. 

\subsection{Model learning}
The model learning is split into three parts: acquiring an {\em object model}; using this object model, with a demonstrated grasp, to build a {\em contact model} for each finger link in contact with the object; and acquiring a {\em hand configuration model} from the demonstrated grasp. After learning the object model can be discarded.

\subsubsection{Object model}
First, a point cloud of the object used for the demonstrated grasp is acquired by a depth camera, from several views. Each point is augmented with the estimated principal curvatures at that point and a surface normal. Thus, the $j^{th}$ point  in the cloud gives rise to a feature $x_j=(p_j, q_j, r_j)$, with the components being its position $p_j \in \mathbb R^3$, orientation $q_j \in SO(3)$ and principal curvatures $r_j=(r_{j,1},r_{j,2}) \in \mathbb R^2$. The orientation $q_j$ is defined by $k_{j,1},k_{j,2}$, which are the directions of the principal curvatures.  For later convenience we use $v=(p,q)$ to denote position and orientation combined. These features $x_j$ allow the object model to be defined as a kernel density estimate of the joint density over $v$ and $r$.
\begin{equation}
\om(v, r) \equiv \pdf^\om(v, r) \simeq \sum_{j=1}^{K_O} w_j \mathcal{K}(v, r|{x_j}, \sigma_{x})
%RD: mu and sigma are not properly defined.
\label{eq:om}
\end{equation}
where $\om$ is short for $\pdf^\om$, bandwidth $\sigma_{x} = (\sigma_{p}, \sigma _{q}, \sigma_{r})$, $K_O$ is the number of features $x_j$ in the object model, all weights are equal $w_j = 1/{K_O}$, and $\mathcal{K}$ is defined as a product:
\begin{equation}\label{eq:kernel_in_se3}
\mathcal{K}(x | \mu, \sigma) = \mathcal{N}_3(p| \mu_p, \sigma_p) \Theta(q| \mu_q, \sigma_q) \mathcal{N}_2(r| \mu_r, \sigma_r)
\end{equation}
where $\mu$ is the kernel mean point, $\sigma$ is the kernel bandwidth, $\mathcal{N}_n$ is an $n$-variate isotropic Gaussian kernel, and ${\Theta}$ corresponds to a pair of antipodal von Mises-Fisher distributions.
\begin{figure*}[t]
\includegraphics[width=\textwidth]{images/training-examples}
%\includegraphics[width=0.1\textwidth]{images/contact-viewall2}
%\includegraphics[width=0.1\textwidth]{images/contact-viewall3}
%\includegraphics[width=0.1\textwidth]{images/contact-viewall4}
%\includegraphics[width=0.1\textwidth]{images/contact-viewall5}
%\includegraphics[width=0.1\textwidth]{images/contact-viewall6}
%\includegraphics[width=0.1\textwidth]{images/contact-viewall7}
%\includegraphics[width=0.1\textwidth]{images/contact-viewall8}
%\includegraphics[width=0.1\textwidth]{images/contact-viewall9}
%\includegraphics[width=0.1\textwidth]{images/contact-viewall10}
\caption{The ten training grasps for the generative model. The final hand pose is shown in yellow, the sensed point cloud in black, and the parts of the point cloud that contribute to each contact model are coloured by the associated link. \label{fig:generative-training}}
\end{figure*}
\subsubsection{Contact models}
When a grasp is demonstrated the final hand pose is recorded. This is used to find all the finger links $L$ and surface features $x_j$ that are in close proximity. A contact model $M_i$ is built for each finger link $i$. Each feature in the object model that is within some distance $\delta_i$ of finger link $L_i$ contributes to the contact model $\cm_i$ for that link. This contact model is defined for finger link $i$ as follows:
\begin{equation}
\cm_i(u, r) \equiv \pdf^\cm_i(u, r) \simeq \frac{1}{Z} \sum_{j=1}^{K_{M_i}} w_{ij} \mathcal{K}(u, r | {x_j}, \sigma_{x})
%RD: mu and sigma are not properly defined.
\label{eq:cm}
\end{equation}
where $u$ is the pose of $\rl_i$ relative to the pose $v_j$ of the $j^{\mathnormal{th}}$ surface feature, $K_{M_i}$ is the number of surface features in the neighbourhood of link $L_i$, $Z$ is the normalising constant, and $w_{ij}$ is a weight that falls off exponentially as the distance between the feature $x_j$ and the closest point $a_{ij}$ on finger link $L_i$ increases:
\begin{equation}
w_{ij} = \begin{cases}\exp(-\lambda ||p_j-a_{ij}||^2) \quad &\textnormal{ if } ||p_j-a_{ij}|| < \delta_i\\
0 \quad &\textnormal{ otherwise},\end{cases}
\label{eq:learning.modeldist.wgh}
\end{equation}
The key property of a contact model is that it is conditioned on local surface features likely to be found on other objects, so that the grasp can be transferred. We use the principal curvatures $r$, but many local surface descriptors would do. %A contact model can be visualised by marginalising out the dimensions for the rigid body transformation $u$, showing us the distribution over the local curvatures that finger link $L_i$ experienced in the demonstrated grasp. 
%
%\begin{figure*}
%\includegraphics[height=2cm]{images/contact-model-learning/handle-grasp}
%\includegraphics[height=2cm]{images/contact-model-learning/link8}
%\includegraphics[height=2cm]{images/contact-model-learning/handle_model_08_00778r}
%\includegraphics[height=2cm]{images/contact-model-learning/link15}
%\includegraphics[height=2cm]{images/contact-model-learning/handle_model_15_00329r}
%  \caption{A training grasp and some contact models arising from it.}
%  \label{fig:contactModels}
%\end{figure*}

\subsection{Hand configuration model}
In addition to a contact model for each finger-link, a model of the hand configuration $h_c \in \mathbb R^D$ is recorded, where $D$ is the number of DoF in the hand. $h_c$  is recorded for several points on the demonstrated grasp trajectory as the hand closed. The learned model is:
\begin{equation}
\hc(h_c) \equiv \sum_{\gamma \in [-\beta, \beta]} w({h_c(\gamma)}) \mathcal{N}_D(h_c|h_c(\gamma), \sigma_{h_c}) 
\label{eq:hc}
\end{equation}
where $w({h_c(\gamma)}) = \exp(-\alpha \|h_c(\gamma) - h^g_c \|^2)$; $\gamma$ is a parameter that interpolates between the beginning ($h^t_c$) and end ($h^g_c$) points on the trajectory, governed via \eq\ref{eq:learning.configmodel.config} below; and $\beta$ is a parameter that allows extrapolation of the hand configuration.
\begin{equation}
h_c(\gamma) = (1 - \gamma)h^g_c + \gamma h^t_c
\label{eq:learning.configmodel.config}
\end{equation}
\subsection{Grasp Transfer}
When presented with a new object $o_{new}$ the contact models must be transferred to that object. A partial point cloud of $o_{new}$ is acquired (from a single view) and recast as a density, $\om_{new}$, again using \eq \ref{eq:om}. The transfer of each contact model $\cm_i$ is achieved by convolving $\cm_i$ with $\om_{new}$. This convolution is approximated with a Monte-Carlo method, resulting in an kernel density model of the pose $s$ of the finger link $i$ (in workspace coordinates) for the new object. The Monte-Carlo procedure samples poses for link $L_i$ on the new object. The $j^{th}$ sample is $\hat{s}_{ij}=(\hat{p}_{ij},\hat{q}_{ij})$. Each sample $\hat{s}_{ij}$ is weighted $w_{ij}$ by its likelihood. These samples are used to build what we term the query density:
\begin{equation}
\qd_i(s) \simeq \sum^{K_{Q_i}}_{j=1} w_{ij} \mathcal{N}_3(p|{\hat{p}_{ij}}, \sigma_{p}) \Theta(q|{\hat{q}_{ij}}, \sigma_{q})%, \quad i = 1, ..., N_L
\label{eq:qd.approx}
\end{equation}
where all the weights are normalised, $\sum_j w_{ij} = 1$. A query density is constructed for every contact model and the new object. These query densities, together with the hand configuration model, are then used to generate grasps. Query density computation is fast, taking $<0.5s$  per grasp model.
\begin{figure*}[t]
\begin{center}
  \includegraphics[width=0.85\textwidth]{images/networkArchitecture.pdf}
  \end{center}
  \caption{The evaluative network architecture.}
\label{fig:networkArchitecture}
\end{figure*}
\subsection{Grasp generation}
Candidate grasps may be generated as follows. Select a query density $k$ and take a sample  $s_k \sim \qd_{k}$. Then, take a sample $h_c \sim C$ from the hand configuration model. This pair of samples together define, via the hand kinematics, a complete grasp $h=(h_w,h_c)$, where $h_w$ is the pose of the wrist and $h_c$ is the configuration of the hand. The initial grasp is then improved by stochastic hill-climbing on a product of experts:
\begin{equation}
\argmax{(h_w, h_c)} \hc(h_c) \prod_{\qd_i \in \mathcal{Q}} \qd_i\left(k_{i}^{\mathrm{for}}\left(h_w, h_c\right)\right)
\label{eq:grasping.product}
\end{equation}
This generate and improvement process has periodic pruning steps, in which only the higher likelihood grasps are retained. It can be run many times, thus enabling the generation of many candidate grasps. In addition, a separate generative model can be learned for each demonstrated grasp. Thus, when presented with a new object, each grasp model can be used to generate and improve grasps. We generate and optimise 100 grasps per grasp type. Finally, the many candidate grasps generated from each grasp model can be compared and ranked according to their likelihoods. The product of experts formulation, however, only ensures that the generated grasps have high likelihood according to the model. There is no estimate of the probability that the grasp will succeed. This motivates the dual architecture in this paper. We now turn to the learning method we used to re-rank the grasps according to predicted success probability. 

\subsection{Training Grasps for the Evaluative Model}

For this study, ten example grasps were provided (Figure~\ref{fig:generative-training}). In contrast to \cite{kopicki2015ijrr}, although seven views of each training object were taken, we trained a separate generative model for each view. This led to a total of 70 generative models being learned, one for each grasp-view combination. Because of this view based training, we filter surface normals on the object model so that for each contact model we only consider points on the object surface with surface normals within +/- 90 degrees of the surface of the finger link. %Second, rather than globally selecting the best grasps regardless of the training grasp type---as in \cite{kopicki2015ijrr}---we select half globally and half we force to be evenly spread across the grasp types. This keeps a broad range of grasp options open for evaluation.
 \label{section:generative}

\section{Improved Generative Learning}

%Describe, in new words, the method from IJRR.
%% This section describes the improved generative method. Only the differences between the original method (given in the previous section) and the new methhod are explained here. 

In this paper we also utilised a more advanced generative model, which we refer to as GM2. This model has three features which are different from the base model GM1. As for GM1, these are not a contribution of this paper and are described fully in \cite{kopicki2019ijrr}. For completeness, however, we briefly describe the three differences between GM2 and GM1. 

\subsection{Object View Model}\label{sec:representations.object}
The first difference is that the learning of grasp models is done per view, rather than per grasp. For a training grasp made on an object viewed from seven viewpoints, there will be seven grasp models learned. This enables grasps to generalise better when the testing object to be grasped is thick and is only seen from a single view. The view based models allow a greater role to be played by the hand shape model and this enables generated grasps to have fingers which `float' behind a back surface that cannot be seen by the robot.

\subsection{Clustering Contact Models}\label{sec:learning.clustering}

The second innovation is the ability to merge grasp models learned from different grasps. In the memory based scheme of GM1, the number of contact models $N_{\mathcal{M}}$ equals the product of the number of training grasps by the number of views. This has two undesirable properties. First, it means that generation of grasps for test objects rises linearly in the number of training grasps. Second, it limits the generalisation power of the contact models. We can overcome these problems by clustering the contact models from each training grasp. To do this we need a measure of the similarity between any pair of contact models. Recall that our contact models are probability densities represented as kernel density estimators. Thus, we need a distance metric in the space of probability densities of a given dimension.

One possibility is to employ Jensen-Shannon distance, but this is slow to evaluate. We therefore start by devising a simple and quick to compute asymmetric divergence. We then build on top of it a symmetric distance. Having obtained this distance measure we can employ our clustering method of choice, which in our case was affinity propagation \cite{frey2007clustering}. After clustering, we compute a cluster prototype as described in \cite{kopicki2019}.

\subsection{Improved Grasp Transfer and Inference}
GM2 utilises the same distance measure to transfer grasps when creating the query densities and also to evaluate candidate grasps. This has the effect of making the proposed grasps more conservative and thus closer to the demonstrated grasps in terms of the type of contacts made with the target object.

We now proceed to describe how we use these models to generate a data-set of 2 million simulated dexterous grasps. \label{section:generative_new}

\section*{References}

%%PLS. NOTE TEMPLATE HERE SHOWN FOR DIFFERENT STYLES FOR 
%%TEXT IN JOURNALS, BOOKS, REVIEW VOLUME AND PROCS. ETC.
%%AT ACTUAL WORK PLS. TYPE WITHOUT THE \myhead COMMAND. 
%%YOU CAN FOLLOW MY EXAMPLE FOR THE COMMAND OF 
%%\thebibliography AFTER THE \end{document} COMMAND.

\vspace*{-5pt}   %ONLY NECESSARY

\myhead{Journal paper}

%1.
\myitem G. Capi, Y. Nasu, L. Barolli, K. Mitobe and K. Takeda,  
Application of genetic algorithms for biped robot gait synthesis 
optimization during walking and going up stairs, 
{\it Advanced Robotics} {\bf 15}(6) (2001) 675--694.

\myhead{Authored book}

%2.
\myitem U. Nehmzow, {\it Mobile Robotics: A Practical 
Introduction}, 2nd edn. (Springer-Verlag, New York, 2003), pp.~25--27.
  
\myhead{Edited book}

%3.
\myitem G. A. Carpenter and S. Grossberg (eds.),  
{\it Neural Networks for Vision and Image Processing} 
(MIT Press, Cambridge, Massachusetts, 1992), pp.~45--60.  

\myhead{Review volume}

%4.
\myitem M. H. Chignell and P. A. Hancock, Horn clause 
representations in human-machine systems with adaptive control, 
in {\it Trends in Ergonomics/Human Factors III}, 
ed.~W. Karwowski (Elsevier, Amsterdam, 1986), p.~76.  

\myhead{Series book}

%5.
\myitem R. M. H. Cheng and R. Rajagopalan, Binary-camera vision  
for guidance control of the automated guided vehicle, in 
{\it Recent Trends in Mobile Robots}, 
Series in Robotics and Automated Systems, Vol.~11 
(World Scientific, New Jersey, 1994), p.~83.

\myhead{Proceedings}

%6.
\myitem A. Dasgupta and Y. Nakamura, Making feasible walking 
motion of humanoid robots from human motion capture data, in 
{\it IEEE Int. Conf. Robotics and Automation (ICRA)} 
(IEEE Press, Detroit, USA, 1999), pp. 1044--1049.
\eject

%FIRST AUTHOR
%\vspace*{13pt}

\vfill\eject

\end{document}

%%BELOW HERE ARE THE SAME BIBLIOGRAPHY TEXT AS ABOVE.
%%PLS. USE THIS COMMAND FOR ACTUAL WORK.
\begin{thebibliography}{0}

\bibitem{1.} G. Capi, Y. Nasu, L. Barolli, K. Mitobe and K. Takeda, 
Application of genetic algorithms for biped robot gait synthesis 
optimization during walking and going up stairs, 
{\it Advanced Robotics} {\bf 15}(6), 675--694 (2001).

\bibitem{2.} U. Nehmzow, {\it Mobile Robotics: A Practical 
Introduction}, 2nd edn. (Springer-Verlag, New York, 2003), pp.~25--27.
  
\bibitem{3.} G. A. Carpenter and S. Grossberg (eds.),  
{\it Neural Networks for Vision and Image Processing} 
(MIT Press, Cambridge, Massachusetts, 1992), pp.~45--60.  

\bibitem{4.} M. H. Chignell and P.A. Hancock, Horn clause 
representations in human-machine systems with adaptive control, 
in {\it Trends in Ergonomics/Human Factors III}, 
ed.~W. Karwowski (Elsevier, Amsterdam, 1986), p.~76.  

\bibitem{5.} R. M. H. Cheng and R. Rajagopalan, Binary-camera vision  
for guidance control of the automated guided vehicle, in 
{\it Recent Trends in Mobile Robots}, 
Series in Robotics and Automated Systems, Vol.~11 
(World Scientific, New Jersey, 1994), p.~83.

\bibitem{6.} A. Dasgupta and Y. Nakamura, Making feasible walking 
motion of humanoid robots from human motion capture data, in 
{\it IEEE Int. Conf. Robotics and Automation (ICRA)} 
(IEEE Press, Detroit, USA, 1999), pp. 1044-1049.

\end{thebibliography}

\end{document}

