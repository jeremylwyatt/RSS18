\documentclass[conference]{IEEEtran}
\usepackage{times}
\usepackage{graphicx}

% numbers option provides compact numerical references in the text. 
\usepackage[numbers]{natbib}
\usepackage{multicol}
\usepackage{multirow}
\usepackage[bookmarks=true]{hyperref}
\usepackage{amsthm,amssymb,amsmath}

\newcommand{\eq}{Eq.~}
\newcommand{\fig}{Fig.~}
\newcommand{\tab}{Tab.~}
\newcommand{\sect}{Sec.~}

\newcommand{\re}{R}
\newcommand{\rf}{R}
\newcommand{\rl}{L}
\newcommand{\om}{O}
\newcommand{\cm}{M}
\newcommand{\fm}{F}
\newcommand{\hc}{C}
\newcommand{\qd}{Q}
\newcommand{\gpm}{T}
\newcommand{\coll}{W}
\newcommand{\pdf}{\mathbf{pdf}}

\newcommand{\argmax}[1]{\underset{#1}{\operatorname{argmax}}\medspace}

\pdfinfo{
   /Author (Homer Simpson)
   /Title  (Robots: Our new overlords)
   /CreationDate (D:20101201120000)
   /Subject (Robots)
   /Keywords (Robots;Overlords)
}

\begin{document}

% paper title
\title{Dexterous Grasping of Novel Objects from a Single View with a Generative-Evaluative Learner}

% You will get a Paper-ID when submitting a pdf file to the conference system
\author{Author Names Omitted for Anonymous Review. Paper-ID [add your ID here]}

%\author{\authorblockN{Michael Shell}
%\authorblockA{School of Electrical and\\Computer Engineering\\
%Georgia Institute of Technology\\
%Atlanta, Georgia 30332--0250\\
%Email: mshell@ece.gatech.edu}
%\and
%\authorblockN{Homer Simpson}
%\authorblockA{Twentieth Century Fox\\
%Springfield, USA\\
%Email: homer@thesimpsons.com}
%\and
%\authorblockN{James Kirk\\ and Montgomery Scott}
%\authorblockA{Starfleet Academy\\
%San Francisco, California 96678-2391\\
%Telephone: (800) 555--1212\\
%Fax: (888) 555--1212}}


% avoiding spaces at the end of the author lines is not a problem with
% conference papers because we don't use \thanks or \IEEEmembership


% for over three affiliations, or if they all won't fit within the width
% of the page, use this alternative format:
% 
%\author{\authorblockN{Michael Shell\authorrefmark{1},
%Homer Simpson\authorrefmark{2},
%James Kirk\authorrefmark{3}, 
%Montgomery Scott\authorrefmark{3} and
%Eldon Tyrell\authorrefmark{4}}
%\authorblockA{\authorrefmark{1}School of Electrical and Computer Engineering\\
%Georgia Institute of Technology,
%Atlanta, Georgia 30332--0250\\ Email: mshell@ece.gatech.edu}
%\authorblockA{\authorrefmark{2}Twentieth Century Fox, Springfield, USA\\
%Email: homer@thesimpsons.com}
%\authorblockA{\authorrefmark{3}Starfleet Academy, San Francisco, California 96678-2391\\
%Telephone: (800) 555--1212, Fax: (888) 555--1212}
%\authorblockA{\authorrefmark{4}Tyrell Inc., 123 Replicant Street, Los Angeles, California 90210--4321}}


\maketitle

\begin{abstract}
Dexterous grasping of a novel object given a single view is an open problem. This paper makes several contributions to the solution of this problem. First, we present a data set of more than two million simulated dexterous grasps for 294 objects drawn from 20 categories. Second, we present a basic architecture for generation and evaluation of dexterous grasps that is trained on a portion of this data set. Third, we train and evaluate eight different architectural variants on this data set. Finally, we present a real robot implementation and evaluate the most promising methods on data set of real object-pose pairs. We show that our best architectural variant achieves a success rate of 87.8\% on novel objects seen from a single view. This improves on the state of the art. 

%Deep neural networks (DNNs) have been used to learn evaluative models (EMs). Given a grasp and an image, an EM indicates the probability of grasp success. Finding a grasp is then an optimisation problem on this evaluation function, searching over the grasp configuration space. This works well for pinch grasps, but it is an open question whether this approach generalises well to dexterous grasps, where the configuration space is of high dimension. An alternative is to learn a generative model (GM), which maps from images to grasps, subsequently refined by search. Factored GMs scale to high-dimensional configuration spaces, allow data-efficient learning, and generate a wide variety of grasp types, fully exploiting the possibilities of dexterous hands. But they give no guarantee as to the probability of grasp success. This paper shows the benefits of a hybrid architecture. It presents and compares multiple versions of three architectures for dexterous grasping: i) pure GM; ii) pure EM and iii) hybrid GM-EM. Extensive empirical studies were executed in both simulation and on a real robot. These show that hybrid GM-EM outperforms pure GM,  which in turn outperforms pure EM. The best performing GM-EM model achieves 87.7\% on a real robot dexterously grasping 49 novel objects in challenging poses.


%A generative-evaluative learning architecture (GEA) is presented. The generative model (GM) is acquired by data efficient learning from demonstration (LfD), and the evaluative model (EM) is trained in simulation, using grasps proposed by the generative model. When a novel object is presented the generative model proposes grasps, which are ranked by the evaluative model. Experiments show that this GEA architecture improves on a pure generative model (GM). On a challenging set of 49 real objects, GEA has a grasp success rate of 77.6\% relative to a pure GM (57.1\%). It is also shown that grasp optimisation using the EM fails to improve on grasps suggested by the GEA, worsening grasp success rate in simulation by 4.8\% against the baseline. These results provide support for generative-evaluative learning  for dexterous grasping.
\end{abstract}

\IEEEpeerreviewmaketitle

\section{Introduction}
% * <jeremy.l.wyatt@gmail.com> 2018-01-29T09:05:33.799Z:
% 
% Dexterous grasping of novel objects. Learning from demonstration. Data efficiency and grasp transfer. Problem of not knowing whether a grasp is likely to succeed or fail. Note that deep grasping methods predict success from global information, but that they require numbers of training examples infeasible for current dexterous hands. Summarise technical contributions. Summarise overall architecture.
% 
% ^.

If robots are to be widely deployed in human populated environments then they must deal with unfamiliar situations. An example is the case of grasping and manipulation. Humans grasp and manipulate hundreds of objects each day, many of which are previously unseen. Yet humans are able to dexterously grasp these novel objects with a rich variety of grasps. In addition, we do so from only a single, brief, view of each object. To operate in our world, dexterous robots must replicate this ability.

This is the motivation for the problem tackled in this paper, which is planning of (i) a dexterous grasp, (ii) for a novel object, (iii) given a single view of that object. We define dexterous as meaning that the robot employs a variety of dexterous grasp types across a set of objects. The combination of constraints (i)-(iii) makes grasp planning hard because surface reconstruction will be partial, yet this cannot be compensated for by estimating pose for a known object model. The novelty of the object, together with incomplete surface reconstruction, and uncertainty about object mass and coefficients of friction, renders infeasible the use of grasp planners which employ classical mechanics to predict grasp quality. Instead, we must employ a learning approach.

This in turn raises the question as to how we architect the learner. Grasp planning comprises two problems: generation and evaluation. Candidate grasps must first be generated according to some distribution conditioned on sensed data. Then each candidate grasp must be evaluated, so as to produce a grasp quality measure (e.g maximum resistable wrench), the probability of grasp success, the likely in-hand slip or rotation, etcetera. These measures are then used to rank grasps so as to select one to execute.
\begin{figure}[t]
\begin{center}
  \includegraphics[width=\columnwidth]{images/contribution.pdf}
  \end{center}
  \caption{The basic architecture of a generative-evaluative learner. When shown a novel object the learned generative model (GM) produces many grasps according to its likelihood model. These are then each evaluated by a learned evaluative model (EM), which predicts the probability of grasp success. The grasps are then re-ranked according to the predicted success probability and the top ranked grasp is executed.}
\label{fig:systemArchitecture}
\end{figure}
Either or both a {\em generative} or {\em evaluative} model may be learned. If only a generative model is learned then evaluation must be carried out using mechanically informed reasoning, which, as we noted, cannot easily be applied to the case of novel objects seen from a single view. If only an evaluative model is learned then grasp generation must proceed by search. This is challenging for true dexterous grasping as the hand may have between nine and twenty actuated degrees of freedom. Thus, for dexterous grasping of novel objects from a single view, it becomes appealing to {\em learn} both the generative and the evaluative model. 

The contributions of this paper are as follows. First, we present a data-set of two million dexterous grasps in simulation that may be used to evaluate dexterous grasping algorithms. Second, we release the source code of the dexterous grasp simulator, which can be used to visualise and re-run the dataset, and gather new data.\footnote{The code and simulated grasp dataset are available \href{https://rusen.github.io/DDG/}{here}. The web page explains how to download the dataset, install physics simulator and re-run the grasps in simulation. The simulator acts as a client alongside a simple http web server to gather new grasp data in a distributed setup (coming soon). Furthermore, we provide a tool to save video-data and replay grasps without running them.} Third, we present a generative-evaluative architecture that combines data efficient learning of the generative model with data intensive learning in simulation of an evaluative model. Fourth, we present multiple variations of the evaluative model. Fifth, we present an extensive evaluation of all these models on our simulated data set. Finally, we compare the two most promising variants on a real robot with a data-set of objects in challenging poses.

The model variants are organised in three dimensions. First, we employ two different generative models (GM1 \cite{kopicki2015ijrr} and GM2 \cite{kopicki2019ijrr}), one of which (GM2) is designed specifically for single view grasping. Second, we employ two different back-bones for the evaluative model based on VGG-16 and ResNet-50. Third, we experiment with two optimisation techniques--gradient ascent (GA) and stochastic annealing (SA)--to search for better grasps using the evaluative model as an objective function.

%The first contribution of this paper is the first generative-evaluative architecture where both generative and evaluative models are learned. Second, three novel evaluative networks, based on existing VGG-16 and ResNet-50 architectures, were proposed. Finally, a grasp simulation dataset containing 2M+ grasps on 295 objects, along with the simulator code will be released to to the community. In order to simulate the wide range of physical variations that real world objects have, a data set is acquired by simulation of  grasps proposed by the generative models on novel objects. We vary the physical characteristics of the objects in each simulated scene (mass, friction) which the robot can not observe. This dataset was used to train three evaluative neural networks. The evaluative models were then used to rank grasps produced by the generative models (GM1 and GM2) in real robot experiments. 

%The paper builds upon two existing generative grasp models that are learned from a small number of demonstrated grasps using the data-efficient method for LfD. Unlike other approaches to deep grasping, which are restricted to power-grasps, the methods are able to perform a wide variety of grasps, including pinch, rim, power, and handle grasps, and to use additional fingers to provide bracing.

The paper is structured as follows. First, we discuss related work. Second, the basic generative model is described in detail and the main features of the extended generative model are sketched. Third, we describe the design of the grasp simulation, the generation of the data set. Fourth, we describe the different architectures employed for the evaluative model. Fifth, we describe the training of the evaluative model, the optimisation variants for the evaluative model and the simulated experimental study. Finally, we present the real robot study.


\section{Background and Related Work}

% * <jeremy.l.wyatt@gmail.com> 2018-01-29T09:06:15.090Z:
% 
% %Give a good overview of the shape of both dexterous grasping, grasping novel objects, and deep networks for grasping.
% 
% Levine-both simulation and 800k grasping papers will be mentioned here. Our method does not perform any grasp planning. Our contribution is a grasp evaluation method that aims to understand the relationship between the point cloud and grasp parameters from the point of view of the camera. 
% 
% 
% ^.

There are four broad approaches to grasp planning for dexterous hands. First, we may employ analytic mechanics to evaluate grasp quality. Second, we may engineer a mapping from sensing to grasp. Third, we may learn this mapping, such as learning a generative model. Fourth, we may learn a mapping from sensing and a grasp to a grasp success prediction. See \cite{bohg2014data} and  \cite{sahbani2012overview} for recent reviews of data driven and analytic methods respectively.

Analytic approaches use mechanical models to predict grasp outcome \cite{bicchi2000a,Liu2000,Pollard2004,miller2004}. This requires models of both object (mass, mass distribution, shape, and surface friction) and manipulator (kinematics, exertable forces and torques). Several grasp quality metrics can be defined using these~\cite{Ferrari1992,Roa2015,Shimoga1996} under a variety of mechanical assumptions. These have been applied to dexterous grasp planning \cite{Boutselis2014,Gori2014,Hang2014,Rosales2012,Saut2012,ciocarlie2009hand}. The main drawback of analytic approaches is that estimation of object properties is hard. Even a small error in estimated shape, friction or mass will render a grasp unstable \cite{zheng2005a}. There is also evidence that grasp quality metrics are not well correlated with actual grasp success \cite{bekiroglu2011b,kim2013a,goins2014a}. It has been shown that a sampling based grasp planner, using an analytic evaluation model, can achieve grasp success of 53\% on real objects with some parameterized shape uncertainty \cite{li2016dexterous}.

An alternative is learning for robot grasping, which has made steady progress. There are probabilistic machine learning techniques employed for surface estimation for grasping \cite{dragiev2011gaussian}; data efficient methods for learning dexterous grasps from demonstration \cite{ben-amor2012a,kopicki2015ijrr,Osa2018}; logistic regression for classifying grasp features from images \cite{saxena2008a}; extracting generalisable parts for grasping \cite{detry2012a} and for autonomous grasp learning \cite{detry2010a}. Deep learning is a recent approach to grasping. Most work is for two finger grippers e.g. \cite{songiros20}. Approaches either learn an evaluation function for an image-grasp pair \cite{levine16,lenz2015deep,gualtieri2016high,mahler2017dex,pinto2016supersizing,johns2016deep}, learn to predict the grasp parameters \cite{redmon2015real,kumra2017iros} or jointly estimate both \cite{morrison18}. The quantity of real training grasps can be reduced by mixing real and simulated data \cite{bousmalis2017using}. 


\begin{table*}[t]
\centering
\small
\resizebox{\columnwidth}{!}{\begin{tabular}{|l|l|l|l|l|l|l|l|}
\hline
References & \multicolumn{3}{|c|}{Grasp type} & Robot & Clutter & Model & Novel  \\ \cline{2-4}
                   & 2-fing. & $>$2-fing. & $>$2-fing. & results & & free & objects\\ 
                   &            & pow.           & dext. &           &  &        &           \\ \hline
\cite{detry2012a,saxena2008a,detry2010a,lenz2015deep,mahler2017dex,johns2016deep,morrison-RSS-18} & \Checkmark & & & \Checkmark &  & \Checkmark & \Checkmark \\ \hline      
\cite{pinto2016supersizing,bousmalis2017using,levine2017,Gualtieri2016,songiros20} & \Checkmark & & & \Checkmark & \Checkmark & \Checkmark & \Checkmark \\ \hline     
\cite{kappler2015leveraging} &  & \Checkmark & &  &  &  & \Checkmark \\ \hline
\cite{zhou20176dof} &  & \Checkmark & &  &  & \Checkmark & \Checkmark \\ \hline     
\cite{lu2017planning,varley2015generating} &  & \Checkmark & & \Checkmark &  & \Checkmark & \Checkmark \\ \hline    
\cite{ben-amor2012a} &  & &  \Checkmark &  \Checkmark &  &  & \Checkmark \\ \hline    
\cite{veres2017modeling,zhou20176dof,kappler2015leveraging,mandikal2021dexterous} &  & &  \Checkmark &   &  & \Checkmark & \Checkmark \\ \hline 
\cite{kopicki2015ijrr,kopicki2019ijrr,Osa2018} &  & &  \Checkmark &  \Checkmark &  & \Checkmark & \Checkmark \\ \hline 
\cite{arruda2016active} &  & &  \Checkmark &  \Checkmark & \Checkmark & \Checkmark & \Checkmark \\ \hline 
This paper &  & &  \Checkmark &  \Checkmark &  & \Checkmark & \Checkmark \\ \hline 
\end{tabular}}
\caption{Qualitative comparison of grasp learning methods.}
\label{tab:comp-related-work}
\end{table*}

A much smaller number of papers have explored deep learning as a method for dexterous grasping. \cite{lu2017planning,varley2015generating,veres2017modeling,zhou20176dof,kappler2015leveraging}. All of these use simulation to generate the training set for learning. Kappler \cite{kappler2015leveraging} showed the ability of a CNN to predict grasp quality for multi-fingered grasps, but uses complete point clouds as object models and only varies the wrist pose for the pre-grasp position, leaving the finger configurations the same. Varley \cite{varley2015generating} and later Zhou \cite{zhou20176dof} went beyond this by varying the hand pre-shape, and predicting from a single image of the scene. Each of these posed search for the grasp as a pure optimisation problem (using simulated annealing or quasi-Newton methods) on the output of the CNN. They, also, take the approach of learning an evaluative model, and generate candidates for evaluation uninfluenced by prior knowledge. Veres \cite{veres2017modeling}, in contrast, learns a deep generative model. Lu \cite{lu2017planning} learns an evaluative model, and then, given an input image, optimises the inputs that describe the wrist pose and hand pre-shape to this model via gradient ascent, but does not learn a generative model. In addition, the grasps start with a heuristic grasp which is varied within a limited envelope. Mandikal uses reinforcement learning in simulation to learn a wide range of dexterous grasp policies that transfer to unseen objects and which do not require a full mesh \cite{mandikal2021dexterous}. The results are only in simulation and the success rate for unseen objects is 53.1\%.\footnote{Personal correspondence with the authors.} Of the papers on dexterous grasp learning with deep networks only two approaches \cite{varley2015generating,lu2017planning} have been tested on real grasps, with eight and five test objects each, producing success rates of 75\% and 84\% respectively. An key restriction of both of these methods is that they only plan the pre-grasp, not the finger-surface contacts, and are thus limited to power-grasps.

Thus, in each case, either an evaluative model is learned but there is no learned prior over the grasp configuration able to be employed as a generative model; or a generative grasp model is learned, but there is no evaluative model learned to select the grasp. Our technical novelty is thus to bring together a data-efficient method of learning a good generative model with an evaluative model. As with others, we learn the evaluative model from simulation, but the generative model is learned from a small number of demonstrated grasps. Table~\ref{tab:comp-related-work} compares the properties of the learning methods reviewed above against this paper. Most works concern pinch grasping. Of the eight papers on learning methods for dexterous grasping, two \cite{varley2015generating,lu2017planning} are limited to power grasps. Of the remaining six, four have no real robot results \cite{veres2017modeling,zhou20176dof,kappler2015leveraging,mandikal2021dexterous}. Of the remaining four, two we directly build on here, the third being a extension of one of those grasp methods with active vision. Finally, our real robot evaluation is extensive in comparison with competitor works on dexterous grasping, comprising 196 real grasps of 40 different objects.

%n learning to grasp divides into two categories, that we label {\em generative} and {\em evaluative} respectively. We quickly summarise the relationship between this paper and these two.
%
%Generative approaches to grasping include the work of detry et al, kopicki et al, these methods learn distributions of grasps from positive examples. In the case of detry, saxena, these are pinch grasps that are associated with features extracted from monocular or stereo images. Both those pieces of work were restricted to pinch grasps. Both Saxena and Detry require significant training samples, but Detry used real grasps, whereas Saxena used simulation to generate training exemplars. Other approaches use LfD rather than autonomous exploration, as this can significantly reduce the training data required. Peters et al introduced a method for generative grasping by grasp warping, demonstrating an ability to handle new objects of a similar global shape to the training objects. Kopicki et al introduced a factored generative model, showing grasp transfer to novel objects from one example of each grasp type. This is the method than we replicate and build on in our work.
%
%Evaluative approaches to grasping include the work of levine, in which a farm of fourteen robot were used over a two month data gathering period. This data is used to train a neural network that predicts the probability of success of a pinch grasp conditional on an image. In that work the only input is an RGB image, but other approaches to deep grasping tenpas use depth images. However, all these approaches are all data intensive. 
%
%The work that is closest to ours, and the only paper of which we are aware on deep learning applied to dexterous grasping, is that of Hermans et al. 

\section{Data Efficient Learning of a Generative Grasp Model from Demonstration}

%Describe, in new words, the method from IJRR.
This section describes the generative model learning. We employ a method \cite{kopicki2015ijrr}, which learns a generative model of a dexterous grasp from a demonstration (LfD). That paper posed it as the problem of learning a factored probabilistic model. The method is split into a model learning phase, a model transfer phase, and the grasp generation phase. 

\subsection{Model learning}
The model learning is split into three parts: acquiring an {\em object model}; using this object model, with a demonstrated grasp, to build a {\em contact model} for each finger link in contact with the object; and acquiring a {\em hand configuration model} from the demonstrated grasp. After learning the object model can be discarded.

\subsubsection{Object model}
First, a point cloud of the object used for the demonstrated grasp is acquired by a depth camera, from several views. Each point is augmented with the estimated principal curvatures at that point and a surface normal. Thus, the $j^{th}$ point  in the cloud gives rise to a feature $x_j=(p_j, q_j, r_j)$, with the components being its position $p_j \in \mathbb R^3$, orientation $q_j \in SO(3)$ and principal curvatures $r_j=(r_{j,1},r_{j,2}) \in \mathbb R^2$. The orientation $q_j$ is defined by $k_{j,1},k_{j,2}$, which are the directions of the principal curvatures.  For later convenience we use $v=(p,q)$ to denote position and orientation combined. These features $x_j$ allow the object model to be defined as a kernel density estimate of the joint density over $v$ and $r$.
\begin{equation}
\om(v, r) \equiv \pdf^\om(v, r) \simeq \sum_{j=1}^{K_O} w_j \mathcal{K}(v, r|{x_j}, \sigma_{x})
%RD: mu and sigma are not properly defined.
\label{eq:om}
\end{equation}
where $\om$ is short for $\pdf^\om$, bandwidth $\sigma_{x} = (\sigma_{p}, \sigma _{q}, \sigma_{r})$, $K_O$ is the number of features $x_j$ in the object model, all weights are equal $w_j = 1/{K_O}$, and $\mathcal{K}$ is defined as a product:
\begin{equation}\label{eq:kernel_in_se3}
\mathcal{K}(x | \mu, \sigma) = \mathcal{N}_3(p| \mu_p, \sigma_p) \Theta(q| \mu_q, \sigma_q) \mathcal{N}_2(r| \mu_r, \sigma_r)
\end{equation}
where $\mu$ is the kernel mean point, $\sigma$ is the kernel bandwidth, $\mathcal{N}_n$ is an $n$-variate isotropic Gaussian kernel, and ${\Theta}$ corresponds to a pair of antipodal von Mises-Fisher distributions.
\begin{figure*}[t]
\includegraphics[width=\textwidth]{images/training-examples}
%\includegraphics[width=0.1\textwidth]{images/contact-viewall2}
%\includegraphics[width=0.1\textwidth]{images/contact-viewall3}
%\includegraphics[width=0.1\textwidth]{images/contact-viewall4}
%\includegraphics[width=0.1\textwidth]{images/contact-viewall5}
%\includegraphics[width=0.1\textwidth]{images/contact-viewall6}
%\includegraphics[width=0.1\textwidth]{images/contact-viewall7}
%\includegraphics[width=0.1\textwidth]{images/contact-viewall8}
%\includegraphics[width=0.1\textwidth]{images/contact-viewall9}
%\includegraphics[width=0.1\textwidth]{images/contact-viewall10}
\caption{The ten training grasps for the generative model. The final hand pose is shown in yellow, the sensed point cloud in black, and the parts of the point cloud that contribute to each contact model are coloured by the associated link. \label{fig:generative-training}}
\end{figure*}
\subsubsection{Contact models}
When a grasp is demonstrated the final hand pose is recorded. This is used to find all the finger links $L$ and surface features $x_j$ that are in close proximity. A contact model $M_i$ is built for each finger link $i$. Each feature in the object model that is within some distance $\delta_i$ of finger link $L_i$ contributes to the contact model $\cm_i$ for that link. This contact model is defined for finger link $i$ as follows:
\begin{equation}
\cm_i(u, r) \equiv \pdf^\cm_i(u, r) \simeq \frac{1}{Z} \sum_{j=1}^{K_{M_i}} w_{ij} \mathcal{K}(u, r | {x_j}, \sigma_{x})
%RD: mu and sigma are not properly defined.
\label{eq:cm}
\end{equation}
where $u$ is the pose of $\rl_i$ relative to the pose $v_j$ of the $j^{\mathnormal{th}}$ surface feature, $K_{M_i}$ is the number of surface features in the neighbourhood of link $L_i$, $Z$ is the normalising constant, and $w_{ij}$ is a weight that falls off exponentially as the distance between the feature $x_j$ and the closest point $a_{ij}$ on finger link $L_i$ increases:
\begin{equation}
w_{ij} = \begin{cases}\exp(-\lambda ||p_j-a_{ij}||^2) \quad &\textnormal{ if } ||p_j-a_{ij}|| < \delta_i\\
0 \quad &\textnormal{ otherwise},\end{cases}
\label{eq:learning.modeldist.wgh}
\end{equation}
The key property of a contact model is that it is conditioned on local surface features likely to be found on other objects, so that the grasp can be transferred. We use the principal curvatures $r$, but many local surface descriptors would do. %A contact model can be visualised by marginalising out the dimensions for the rigid body transformation $u$, showing us the distribution over the local curvatures that finger link $L_i$ experienced in the demonstrated grasp. 
%
%\begin{figure*}
%\includegraphics[height=2cm]{images/contact-model-learning/handle-grasp}
%\includegraphics[height=2cm]{images/contact-model-learning/link8}
%\includegraphics[height=2cm]{images/contact-model-learning/handle_model_08_00778r}
%\includegraphics[height=2cm]{images/contact-model-learning/link15}
%\includegraphics[height=2cm]{images/contact-model-learning/handle_model_15_00329r}
%  \caption{A training grasp and some contact models arising from it.}
%  \label{fig:contactModels}
%\end{figure*}

\subsection{Hand configuration model}
In addition to a contact model for each finger-link, a model of the hand configuration $h_c \in \mathbb R^D$ is recorded, where $D$ is the number of DoF in the hand. $h_c$  is recorded for several points on the demonstrated grasp trajectory as the hand closed. The learned model is:
\begin{equation}
\hc(h_c) \equiv \sum_{\gamma \in [-\beta, \beta]} w({h_c(\gamma)}) \mathcal{N}_D(h_c|h_c(\gamma), \sigma_{h_c}) 
\label{eq:hc}
\end{equation}
where $w({h_c(\gamma)}) = \exp(-\alpha \|h_c(\gamma) - h^g_c \|^2)$; $\gamma$ is a parameter that interpolates between the beginning ($h^t_c$) and end ($h^g_c$) points on the trajectory, governed via \eq\ref{eq:learning.configmodel.config} below; and $\beta$ is a parameter that allows extrapolation of the hand configuration.
\begin{equation}
h_c(\gamma) = (1 - \gamma)h^g_c + \gamma h^t_c
\label{eq:learning.configmodel.config}
\end{equation}
\subsection{Grasp Transfer}
When presented with a new object $o_{new}$ the contact models must be transferred to that object. A partial point cloud of $o_{new}$ is acquired (from a single view) and recast as a density, $\om_{new}$, again using \eq \ref{eq:om}. The transfer of each contact model $\cm_i$ is achieved by convolving $\cm_i$ with $\om_{new}$. This convolution is approximated with a Monte-Carlo method, resulting in an kernel density model of the pose $s$ of the finger link $i$ (in workspace coordinates) for the new object. The Monte-Carlo procedure samples poses for link $L_i$ on the new object. The $j^{th}$ sample is $\hat{s}_{ij}=(\hat{p}_{ij},\hat{q}_{ij})$. Each sample $\hat{s}_{ij}$ is weighted $w_{ij}$ by its likelihood. These samples are used to build what we term the query density:
\begin{equation}
\qd_i(s) \simeq \sum^{K_{Q_i}}_{j=1} w_{ij} \mathcal{N}_3(p|{\hat{p}_{ij}}, \sigma_{p}) \Theta(q|{\hat{q}_{ij}}, \sigma_{q})%, \quad i = 1, ..., N_L
\label{eq:qd.approx}
\end{equation}
where all the weights are normalised, $\sum_j w_{ij} = 1$. A query density is constructed for every contact model and the new object. These query densities, together with the hand configuration model, are then used to generate grasps. Query density computation is fast, taking $<0.5s$  per grasp model.
\begin{figure*}[t]
\begin{center}
  \includegraphics[width=0.85\textwidth]{images/networkArchitecture.pdf}
  \end{center}
  \caption{The evaluative network architecture.}
\label{fig:networkArchitecture}
\end{figure*}
\subsection{Grasp generation}
Candidate grasps may be generated as follows. Select a query density $k$ and take a sample  $s_k \sim \qd_{k}$. Then, take a sample $h_c \sim C$ from the hand configuration model. This pair of samples together define, via the hand kinematics, a complete grasp $h=(h_w,h_c)$, where $h_w$ is the pose of the wrist and $h_c$ is the configuration of the hand. The initial grasp is then improved by stochastic hill-climbing on a product of experts:
\begin{equation}
\argmax{(h_w, h_c)} \hc(h_c) \prod_{\qd_i \in \mathcal{Q}} \qd_i\left(k_{i}^{\mathrm{for}}\left(h_w, h_c\right)\right)
\label{eq:grasping.product}
\end{equation}
This generate and improvement process has periodic pruning steps, in which only the higher likelihood grasps are retained. It can be run many times, thus enabling the generation of many candidate grasps. In addition, a separate generative model can be learned for each demonstrated grasp. Thus, when presented with a new object, each grasp model can be used to generate and improve grasps. We generate and optimise 100 grasps per grasp type. Finally, the many candidate grasps generated from each grasp model can be compared and ranked according to their likelihoods. The product of experts formulation, however, only ensures that the generated grasps have high likelihood according to the model. There is no estimate of the probability that the grasp will succeed. This motivates the dual architecture in this paper. We now turn to the learning method we used to re-rank the grasps according to predicted success probability. 

\subsection{Training Grasps for the Evaluative Model}

For this study, ten example grasps were provided (Figure~\ref{fig:generative-training}). In contrast to \cite{kopicki2015ijrr}, although seven views of each training object were taken, we trained a separate generative model for each view. This led to a total of 70 generative models being learned, one for each grasp-view combination. Because of this view based training, we filter surface normals on the object model so that for each contact model we only consider points on the object surface with surface normals within +/- 90 degrees of the surface of the finger link. %Second, rather than globally selecting the best grasps regardless of the training grasp type---as in \cite{kopicki2015ijrr}---we select half globally and half we force to be evenly spread across the grasp types. This keeps a broad range of grasp options open for evaluation.
 \label{section:generative}

\section{The Evaluative Model} \label{section:learning}

The generative model used to generative grasps, explained in the previous section, works with a partial cloud with no assumptions about the object's weight, model, or friction parameters. In a fashion similar to \cite{Levine1}, it is possible to perform the generated grasps and record the outcome. The collective set of grasps and outcomes in simulation, accompanied with corresponding depth images, can be used to train a neural network that can predict whether new image, grasp pairs will succeed or not. This is the approach employed in this paper. 

It is time-consuming and expensive to collect real grasp data with using robotic arms with dexterous hands. Unlike gripper + arm combinations which require relatively less supervision \cite{Levine1}, dexterous hands can be much more fragile due to their complexity. Advances in physics simulators have made it possible to re-create robotic experiment setups in simulation. We created a simulated experimental setup in order evaluate grasp, which allowed us to collect as much data as needed in a short period of time with no supervision.

In this section, we discuss the architecture of our grasp predictor network $f(I_t, h_t)$, where $I^t$ stands for a colorized depth image of the object, and $h_t = (h_{tw}, h_{tc})$ is the grasp parameters with respect to the camera used to acquire the depth image. The network outputs a grasp success probability between $[0,1]$ for any image $I_t$ and grasp $h_t$ pair. The reader is advised to refer to Section \ref{section:simulation} for details of data collection in simulation.

\begin{figure*}[ht]
  \includegraphics[width=\textwidth]{images/networkArchitecture.pdf}
  \caption{The network architecture for grasp success prediction. First 13 layers of the VGG network has been frozen in order to speed-up the training. Hand shape parameters and visual parameters are added together, and processed through 4 Fully Connected + RELU layer pairs. The outcome of the network is a grasp success probability, obtained from one of the two outputs of the final softmax layer (success and failure probabilities).}
\label{fig:networkArchitecture}
\end{figure*}

Our network architecture is given in Figure~\ref{. The purpose of the network is to learn the relationship between the point cloud, given as a colorized depth image, and grasp (hand shape) parameters which encode the configuration of the hand with respect to the camera frame. This is a complex task, as the kinematic model of the hand is unknown to the network, and it has to consider each grasp as a black box: The network knows the inputs that configure the hand, including the joint positions, and only has access to the outcome in the form of success/failure. Implementation-wise, our network combines a version of VGG-16 network with inherited weights from training on ImageNet, with fully connected layers consisting of 1024 nodes each, followed by RELU layers. 
% Talk more about the architecture

The Depth image colorization employed in our paper is an operation that takes a 1-channel depth image $I_{t}^{depth}$ and produces a 3-channel image $I_t$ that can be directly given as input to the VGG-16 network. First, the $640 \times 480$ depth image is cropped by a square window of size $460 \times 460$, located in the center of the image. Then, it is down-sampled to an image of size $224 \times 224$. The colorization process creates a $224 \times 224$ 3-channel image, where the first two channels of the output image $I_t$ are simplified mean curvature and simplified Gaussian curvature, respectively. The formula of mean curvature is $h = {gr}_{xx} + {gr}_{yy}$, where ${gr}_{xx}$ is the second gradient in horizontal direction in a $1 \times 3$ window, and ${gr}_{yy}$ is its vertical counterpart. Similarly, Gaussian curvature $k = {gr}_{xx} \times {gr}_{yy} - ({gr}_{xy})^2$, where ${gr}_{xy}$ is the second-order gradients with respect to both directions (Is this the right way to put it?).

The output is a grasp success probability, which is encoded by two output nodes which measure success and failure probability, using a softmax layer, as shown in Figure \ref{fig:networkArchitecture}. The VGG-16 network is used in order to extract visual features from the colorized depth image of the scene. The background plane (virtual table) is subtracted from the point cloud of the object, thus a segmented view of the object's point cloud is given to the network. 

We use the cross entropy loss in order to train the network, as shown in Equation \ref{equation:crossentropy}.

\begin{equation}
H_{y'}(y) := - \sum_{i} ({y_i' \log(y_i) + (1-y_i') \log (1-y_i)})
\label{equation:crossentropy}
\end{equation}

where $y = f(I_i, h_i)$ is the predicted grasp success of grasp $h_i$, and $I_i$ is the associated colorized depth image of grasp $h_i$.

We trained our network on around 450000 grasps on 7000 distinct scenes, where each scene contains a random instantiation of one of the objects in the dataset with varying rigid body transformations applied, as well as friction and weight changes. The network was tested on 80000 grasps on 1200 scenes of unseen objects, as well as real robot experiments, as explained in Section \ref{section:experiments}. In order to make a direct comparison with Kopicki et al. \cite{kopicki2015ijrr}, we pick top grasps based on both the original ranking, as explained in Section \ref{section:generative}, and according to the predicted success probabilities by the network. 

We opted to use a grasp success prediction network due to the fact that the grasp generator function, explained in the previous chapter, provides alternatives of most intuitive types of grasps. A logical extension of this work would be to pair our learning algorithm with a grasp generator network, which we consider as future work.

Overall network architecture. VGG summary. Description of new layers. Representation of hand parameters, frames of reference, camera image conversion for VGG, trajectory of wrist and fingers.

\section{Training the Evaluative Model from Simulation} \label{section:simulation}
%Introductory sentences here
In this section, we describe how we generated a realistic simulated data set for dexterous grasping. This captures variations in both observable (e.g. object pose) and unobservable (e.g. surface friction) parameters.

To generate the training set, a simulated depth image of a scene containing a single unfamiliar object is generated. Using either of the generative models GM1 or GM2, grasps are generated and executed in simulation. The success or failure of each simulated grasp is recorded. Producing a good simulation for evaluating grasps is non-trivial. An important problem is that the data set must capture the natural uncertainty in unobservable variables, such as mass and friction. Since many of these parameters are unobservable we are thus creating a data set such that the grasp policy must work across a range of variations. This is thus a form of {\em domain randomisation}. A similar technique has been employed by \cite{mahler2017dex}, but we extend it from a single grasp quality metric to full rigid body simulation.

\subsection{Features and Constraints of the Virtual Environment}
\label{subsection:environment}

The collected 3D model dataset contains 294 objects from 20 classes, namely, bottles, bowls, cans, boxes, cups, mugs, pans, salt and pepper shakers, plates, forks, spoons, spatulas, knives, teapots, teacups, tennis balls, dustpans, scissors, funnels and jugs (Figure \ref{fig:allObjects}). All objects in the dataset can be grasped using the DLR-II hand, although there are limitations on how some object classes can be approached. For example, teapots and jugs are not easy to grasp except by their handles due being larger than the hand's maximum aperture, while small objects such as salt and pepper shakers can be approached in more creative ways. The number of objects in each class varies from 1 (dustpan) to 25 (bottles). Long/thin objects such as kitchen utensils are placed vertically in a short, heavy stand in order to make them graspable without touching the table. This reflects the real-world scenario, as attempting to grasp a spatula lying on a table would be dangerous for the robotic hand. In total, 250 objects from all 20 classes were allocated for training and validation, while the remaining 44 objects from 19 classes belong to the test set.

We employ MuJoCo \cite{MuJoCo} as the rigid-body simulator. Since MuJoCo requires that objects comprise of convex parts, all 294 objects were decomposed into convex parts using V-HACD algorithm \cite{V-HACD}. The number of sub-parts varies from 2 to 120.

During the scene creation, the object is placed on the virtual table at a pseudo-random pose. Most objects are placed in a canonical upright pose, and only randomly rotated around the gravity axis (akin to a turntable). The objects belonging to the mug and cup classes have fully random 3D rotations, as it is possible to grasp them in almost any setting.

\begin{figure}
  \includegraphics[width=\linewidth]{images/allObjects-small.pdf}
  \caption{A sample of the 294 objects from all 20 object classes.
  \label{fig:allObjects}}
\end{figure}

To achieve domain randomisation, prior distributions for mass, size and frictional coefficient were estimated from real-world data. The properties of simulated objects are sampled from these priors. For each object its mean size, mass and friction coefficient are matched to a real counterpart. For each trial, the size is randomly scaled by a factor in the range [0.9,1.1], while remaining within the grasp aperture of the hand. Object mass is uniformly sampled from a category specific range, estimated from real objects (Table~\ref{fig:weights}). The friction coefficient of each object is sampled from a range of $[0.5, 1]$ in MuJoCo default units, intended to simulate surfaces from low-friction (metal) to high-friction (rubber). This variation is critical to ensuring that the evaluative model will predict the robustness of a grasp to unobservable variations.
\begin{table}[]
\centering
\caption{Mass ranges for each object class (grams).}
\label{fig:weights}
\resizebox{\linewidth}{!}{\begin{tabular}{|l|l|l|l|l|l|l|}
\hline
Bottle & Bowl     & Box     & Can     & Cup    & Fork    & Pan     \\ \hline
30-70  & 50-400   & 50-500  & 200-400 & 30-330 & 40-80   & 150-450 \\ \hline
Plate  & Scissors & Shaker  & Spatula & Spoon  & Teacup  & Teapot  \\ \hline
40-80  & 50-150   & 100-160 & 40-80   & 40-80  & 150-250 & 500-800 \\ \hline
Jug    & Knife    & Mug     & Funnel  & Ball   & Dustpan &         \\ \hline
80-200 & 50-150   & 250-350 & 40-80   & 50-70  & 100-150 &         \\ \hline
\end{tabular}}
\end{table}
 
For depth image simulation the Carmine 1.09 depth sensor installed on the robot is simulated with a modified version of the Blensor Kinect sensor simulator \cite{KinectSimulator}. For each object, we vary the camera orientation and distance from the object, as well as object mass, friction, scale, location and orientation. We add a small three-dimensional positional noise to each point in the sensor output to simulate calibration errors.

A 3D mesh-model of the DLR-II hand has been used in the simulator. There are no kinematic constraints on how the hand may grasp an object, other than collisions with the table. To ensure realism, we use impedance control for the hand.
%The reasons for the most critical of these decisions are now given in slightly more detail. First, in order to create a realistic simulation environment, we chose the MuJoCo \cite{MuJoCo} physics simulator over other simulators (OpenSim, BulletPhysics, ODE, NVIDIA PhysX) for two reasons: 
%\begin{itemize}
%\item MuJoCo uses generalized coordinates and optimization-based contact dynamics, resulting in fewer numerical instabilities,
%\item MuJoCo is optimized for the quality of physics as well as its speed, hence improving the quality of the physics simulation.
%\end{itemize}
\begin{figure}
  \includegraphics[width=\linewidth]{images/decomposition.png}
  \caption{Approximate convex decomposition of some objects in our dataset. Best viewed in colour.}
  \label{fig:objectDecomposition}
\end{figure}

Table \ref{fig:graspperf} shows the success rates of the generated grasps in each class, when attempted with the grasps ranked by the Generative Model (GM1). The sampled grasps perform well on a number of classes including Dustpans, Scissors, Spoons, and Mugs. Some objects can only be grasped in certain ways, i.e. not all 10 training grasps are applicable to all objects.

\begin{table}[]
\centering
\caption{The average and \textbf{top} grasp success rates (\%) of GM1 on simulated data.}
\label{fig:graspperf}
\resizebox{\linewidth}{!}{\begin{tabular}{|l|l|l|l|l|l|l|}
\hline
Bottle & Bowl     & Box     & Can     & Cup    & Fork    & Pan     \\ \hline
35 - \textbf{47} & 26 - \textbf{61}   & 16 - \textbf{30}  & 41 - \textbf{92} & 44 - \textbf{59} & 59 - \textbf{68}   & 37 - \textbf{57} \\ \hline
Plate  & Scissors & Shaker  & Spatula & Spoon  & Teacup  & Teapot  \\ \hline
50 - \textbf{95}  & 62 - \textbf{69}   & 47 - \textbf{53} & 57 - \textbf{65}   & 63 - \textbf{82}  & 48 - \textbf{91} & 26 - \textbf{23} \\ \hline
Jug    & Knife    & Mug     & Funnel  & Ball   & Dustpan &         \\ \hline
24 - \textbf{43} & 58 - \textbf{65}   & 40 - \textbf{80} & 52 - \textbf{65}   & 28 - \textbf{82}  & 60 - \textbf{78} & 45 - \textbf{63} \\ \hline
\end{tabular}}
\end{table}

%\begin{table}[]
%\centering
%\caption{Mass ranges for each object class (grams).}
%\label{fig:weights}
%\resizebox{\linewidth}{!}{\begin{tabular}{|l|l|l|l|l|l|l|}
%\hline
%Bottle & Bowl     & Box     & Can     & Cup    & Fork    & Pan     \\ \hline
%35.5 - \textbf{47.7}\% & 26.4 - \textbf{61.2}\%   & 16.5 - \textbf{30.1}\%  & 41.4 - \textbf{92.6}\% & 44.7 - \textbf{59.9}\% & 59.6 - \textbf{68.1}\%   & 37.9 - %\textbf{57.3}\% \\ \hline
%Plate  & Scissors & Shaker  & Spatula & Spoon  & Teacup  & Teapot  \\ \hline
%50.2 - \textbf{95.5}\%  & 62.7 - \textbf{69.9}\%   & 47.3 - \textbf{53.3}\% & 57.4 - \textbf{65.7}\%   & 63.4 - \textbf{82.4}\%  & 48.2 - \textbf{91.2}\% & 26.9 - %\textbf{23.9}\% \\ \hline
%Jug    & Knife    & Mug     & Funnel  & Ball   & Dustpan &         \\ \hline
%24.9 - \textbf{43.9}\% & 58.3 - \textbf{65.0}\%   & 40.7 - \textbf{80.9}\% & 52.3 - \textbf{65.9}\%   & 28.0 - \textbf{82.8}\%  & 60.1 - \textbf{78.8}\% & 45.8 - %\textbf{63.2}\%        \\ \hline
%\end{tabular}}
%\end{table}

\subsection{Data Collection Methodology}
\label{subsection:dataCollection}

The data set is divided into units called \textit{scenes}, where each scene comprises a single object placed on a table. This object has a specific set of physical parameters, as described below. Many views and grasps are attempted per scene. Below, we specify the time flow of data collection:

\begin{enumerate}
\item A novel instance of an object from the dataset is generated and placed on a virtual table. Variations are applied to object pose, scale, mass, and friction coefficients.
\item A simulated camera takes a depth image $I_s$ of the scene, converted to a point cloud $P_s$. The viewpoint ${elevation}_s$ of the view point is from 30-57 degrees. The ${azimuth}_s$ is sampled from $[0, 2\pi]$. 
\item All points in the point cloud $P_s$ are shifted by a three-dimensional vector sampled from a Gaussian distribution with parameters $\mu=0$ and $\sigma = 0.004$ (unit: meter).
\item Given $P_s$, the chosen generative model (GM1 or GM2) proposes the candidate grasps. For GM1 and GM2, we choose up to 10 and 50 top grasps per each one of the 10 training grasps, respectively.
\item The grasps are applied to the object in simulation. Before the execution of each grasp, we run a collision check with the virtual table (without the object). The grasps that fail this test are marked as \textit{collided}.
\item 19 further simulated depth images are taken from other viewpoints around the object, as explained in step 2. Images with fewer than 250 depth points are discarded. We then sample with replacement from the remaining images and associate each sampled image and viewpoint with a grasp created in step 3.
\item The grasp outcome, trajectory and depth image are stored for each trial. The grasp parameters are converted to the camera frame for the associated view.
\end{enumerate}

%Each candidate grasp $h_i = \{w_0, ..., w_{n}\}$ consists of a series of 10 waypoints along : $w_0$, ..., $w_{n}$. A waypoint $w_k$ is a 27-element vector that specifies full configuration of the hand in joint space: 3 dimensions for 3D coordinates and 4 dimensions for the orientation of the wrist, and 20 parameters specifying each finger joint's activation. 
%After a grasp $h_i$ is generated in world coordinates, the waypoints that belong to the grasp are converted to the camera's frame of reference. 
%The goal of our network architecture is to learn which grasps are more likely to succeed given a point cloud, where both input channels are represented in terms of the camera frame of reference. %This point differentiates us from the work of Levine et al. \cite{Levine1}, where camera coordinates are not used. It should be noted that the possible camera locations in our simulated data covers a larger space, with full circular movement $[0, 2\pi]$ on azimuth and $[30-57]$ range in elevation. Our scenes do not have any distinguishing landmarks such as a bin or robot base, which may aid the network in locating the camera in the scene. 

In each scene $S_i$, a number of depth images are taken $\{I_{ik}\}_{k=0}^{20}$, in the manner explained above. The first image $I_{i0}$ is used to generate grasps, as explained in Section \ref{section:generative}. We typically perform 100-500 grasps per scene. Attaching different views to each grasp instead of the seed image $I_{i0}$ ensures there is more variation in terms of viewpoints, resulting in a richer dataset.

Once a grasp is performed in simulation, it is considered a success if an object is lifted one metre above the table, and held there for two seconds. If the object slips from the hand during lifting or holding, the grasp is a failure. 

\begin{figure}[t]
\includegraphics[width=\columnwidth]{images/frictionweight}
%\includegraphics[width=0.24\textwidth]{images/Pan4_2_HFLW}
%\includegraphics[width=0.24\textwidth]{images/Pan4_2_LFLW}
%\includegraphics[width=0.24\textwidth]{images/Pan4_2_HFHW}
%\includegraphics[width=0.24\textwidth]{images/Pan4_2_LFHW}\\
%%\includegraphics[width=0.96\textwidth]{images/key-to-eval-training}\\
%\includegraphics[width=0.24\textwidth]{images/Pan4_HFLW}
%\includegraphics[width=0.24\textwidth]{images/Pan4_LFLW}
%\includegraphics[width=0.24\textwidth]{images/Pan4_HFHW}
%\includegraphics[width=0.24\textwidth]{images/Pan4_LFHW}
\caption{Creating a data set for robust evaluation. (Top row) The same pinch grasp, executed on the same object, with varying friction and mass parameters. (Bottom row) A more robust power grasp, executed on the same object, with the same variation in friction and mass. \label{fig:evaluative-training}}
\end{figure}

Using this method, we generated a data set (DS1) of 1.28 million simulated grasps using GM1 as the generative model and a data set of 1.136 million additional grasps (DS2) using GM2 \footnote{Visit \href{https://rusen.github.io/DDG/}{https://rusen.github.io/DDG} to download the data.}. Each grasp in DS1-test and DS2 can be replayed in MuJoCo and the sets are decomposed for train, validation and test purposes. We give the dataset statistics in Table~\ref{tab:data}. The ratio of successful grasps in the dataset is less than 50\% for GM1, and is more than 50\% for GM2. In order to have a balanced training set, DS1 and DS2 only contain scenes that have at least one successful grasp. During training, the datasets were balanced by under-sampling the failure cases in DS1-Tr and over-sampling the failure cases for DS2-Tr. No balancing was performed for the validation and test sets.
\begin{table*}[t]
\centering
\caption{Statistics of the simulated data sets.}
\label{tab:data}
\begin{tabular}{|l|l|l|l|l|l|l|l|l|l|l|} \hline
Data set & Generative &  Subset & \# Scenes & Top-grasp & Top-grasp & Top grasp & Total & Total  & Total  & Total \\ 
              & Model         &              &                   &  \# succs  & \# fails       & \% succs  & grasps   & \# succs      & \# fails  & \% succs  \\ \hline
 DS1-Tr & GM1 & Train & 17714 & 10100 & 7614 & 57.0\% & 1,058,430 & 479,941 & 578,489 & 45.3\% \\ \hline
 DS1-V  & GM1 & Validate & 2309 & 1290 & 1019 & 55.9\% & 122,944 & 61,256 & 61,688 & 49,8\% \\ \hline
 DS1-Te & GM1& Test & 1539 & 1070 & 469 & 69.5\% & 99,521 & 48,084 & 51,437 & 48.3\% \\ \hline
 DS2-Tr  & GM2 & Train & 5377 & 3771 & 1606 & 70.1\% & 943,481 & 533,282 & 410,199 & 56.5\% \\ \hline
 DS2-V   & GM2 & Validate & 544 & 378 & 166 & 69.4\% & 68,586 & 39,559 & 29,027 & 57.7\% \\ \hline
 DS2-Te  & GM2 & Test & 988 & 781 & 207 & 79.0\% & 124,137 & 73,836 & 50,301 & 59.5\% \\ \hline
\end{tabular}
\end{table*}



\section{Real robot experiment}
\label{section:experiments}

To test our generative-evaluative learning architecture we compared the grasp it proposes to the grasp proposed by the generative learner alone. Since \citet{kopicki2015ijrr} showed a 77.7\% success rate with the original generative algorithm we generated a new test set that contained both more challenging objects and placed them in challenging poses. The difficulty single-view grasping with a depth camera depends greatly on the pose of the object relative to the camera. The set comprised 40 test objects (Figure~\ref{fig:real-objects}) and another six training objects. The training objects were used by the human to demonstrate ten example grasps (Figure~\ref{fig:generative-training}). The 40 test objects were used to generate 49 object-pose pairs. From the 40 objects, 35 belonged to object classes in the simulation dataset, while the remaining five do not. 

\begin{figure}
  \includegraphics[width=\linewidth]{images/objects.jpg}
  \caption{The real objects. The training objects are on the left, testing objects are on the right.}
  \label{fig:real-objects}
\end{figure}

The pure generative model architecture (GM) and the generative-evaluative architecture (GEA) were evaluated using a paired trials methodology. Each was presented with the same object-pose combinations. Each generated a ranked list of grasps, and the highest ranked grasp was executed. The highest-ranked grasp based on the predicted success probability of the network is performed on each scene. 

%Training parameters for network. Training of example grasps for learning from demonstration. Creation of real test data set. Paired comparisons methodology with vanilla LFD algorithm (pose + object + camera view).
%
%The actual grasping tests have been performed on the real robot. 

\section{Conclusion} 
\label{sec:conclusion}

The challenge for dexterous grasping of novel objects from a single view is to quickly generate feasible grasps, and to have some measure of confidence that they will succeed. The way we have addressed this is to use two learned models: one generative, one evaluative. First, we learn a generative model using a highly data efficient method for learning from demonstration. This learns a good prior over feasible grasps in a way that can be transferred to novel objects, from just ten exemplars. This generative model is then used to generate grasps for novel objects, tested in simulation, the results of which are used to train an evaluative model with a data intensive learning algorithm. By testing simulated grasps across unobservable parameters such as friction and mass we can create an evaluative model that is robust to unobservable variation. We showed experimentally that our generative-evalutive architecture  improves the grasp success rate on real novel objects from substantially, and this difference is statistically significant at 0.05. 

%\section*{Acknowledgments}

%% Use plainnat to work nicely with natbib. 

%\section{RSS citations}
%
%Please make sure to include \verb!natbib.sty! and to use the
%\verb!plainnat.bst! bibliography style. \verb!natbib! provides additional
%citation commands, most usefully \verb!\citet!. For example, rather than the
%awkward construction 
%
%{\small
%\begin{verbatim}
%\cite{kalman1960new} demonstrated...
%\end{verbatim}
%}
%
%\noindent
%rendered as ``\cite{kalman1960new} demonstrated...,''
%or the
%inconvenient 
%
%{\small
%\begin{verbatim}
%Kalman \cite{kalman1960new} 
%demonstrated...
%\end{verbatim}
%}
%
%\noindent
%rendered as 
%``Kalman \cite{kalman1960new} demonstrated...'', 
%one can
%write 
%
%{\small
%\begin{verbatim}
%\citet{kalman1960new} demonstrated... 
%\end{verbatim}
%}
%\noindent
%which renders as ``\citet{kalman1960new} demonstrated...'' and is 
%both easy to write and much easier to read.
%  
%\subsection{RSS Hyperlinks}
%
%This year, we would like to use the ability of PDF viewers to interpret
%hyperlinks, specifically to allow each reference in the bibliography to be a
%link to an online version of the reference. 
%As an example, if you were to cite ``Passive Dynamic Walking''
%\cite{McGeer01041990}, the entry in the bibtex would read:
%
%{\small
%\begin{verbatim}
%@article{McGeer01041990,
%  author = {McGeer, Tad}, 
%  title = {\href{http://ijr.sagepub.com/content/9/2/62.abstract}{Passive Dynamic Walking}}, 
%  volume = {9}, 
%  number = {2}, 
%  pages = {62-82}, 
%  year = {1990}, 
%  doi = {10.1177/027836499000900206}, 
%  URL = {http://ijr.sagepub.com/content/9/2/62.abstract}, 
%  eprint = {http://ijr.sagepub.com/content/9/2/62.full.pdf+html}, 
%  journal = {The International Journal of Robotics Research}
%}
%\end{verbatim}
%}
%\noindent
%and the entry in the compiled PDF would look like:
%
%\def\tmplabel#1{[#1]}
%
%\begin{enumerate}
%\item[\tmplabel{1}] Tad McGeer. \href{http://ijr.sagepub.com/content/9/2/62.abstract}{Passive Dynamic
%Walking}. {\em The International Journal of Robotics Research}, 9(2):62--82,
%1990.
%\end{enumerate}
%%
%where the title of the article is a link that takes you to the article on IJRR's website. 
%
%
%Linking cited articles will not always be possible, especially for
%older articles. There are also often several versions of papers
%online: authors are free to decide what to use as the link destination
%yet we strongly encourage to link to archival or publisher sites
%(such as IEEE Xplore or Sage Journals).  We encourage all authors to use this feature to
%the extent possible.


\bibliographystyle{plainnat}
\bibliography{deep-grasping,references,main}

\end{document}


